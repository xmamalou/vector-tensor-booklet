\documentclass[12pt]{article}

\usepackage{geometry} % for margin and padding
\usepackage{fancyhdr} % for footer and header
\usepackage[center]{titlesec}
\usepackage[export]{adjustbox} 
\usepackage{amsmath} % for math. Use with `align` or `align*` environment
\usepackage{amsfonts}
\usepackage{mathrsfs}
\usepackage{amssymb}
\usepackage{amsthm}
\usepackage{tikz}
\usepackage{tensor}
% uncomment to use package for circuit design
%\usepackage{circuitikz}

% this sets up fonts, the font used is JetBrains Mono
\usepackage{fontspec}

\defaultfontfeatures{Mapping=tex-text,Scale=MatchLowercase}
\setmainfont[
	Path		= ../res/fonts/,
	Extension	= .ttf,
	Ligatures 	= TeX,
	BoldFont	= JetBrains Mono-Bold,
	ItalicFont 	= JetBrains Mono-Italic,
	BoldItalicFont	= JetBrains Mono-BoldItalic
]{JetBrains Mono-Regular}

\setlength{\columnsep}{4cm}

% instead of using a big file for a paper, use instead smaller chunks.
\usepackage{subfiles}

% general formatting settings
\tolerance=1
\emergencystretch=\maxdimen
\hyphenpenalty=10000
\hbadness=10000

\newgeometry{vmargin={15mm}, hmargin={13mm,13mm}}

\newtheorem{definition}{Ορισμός}[subsection]
\newtheorem{theorem}{Θεώρημα}[subsection]
\newtheorem{corollary}{Πόρισμα}[subsection]
\newtheorem*{remark}{Σημείωση}

\renewcommand\qedsymbol{$\blacksquare}

\newtheorem*{unnum}{Θεώρημα}

\newcommand{\dist}{\operatorname{dist}}
\newcommand{\intr}{\operatorname{int}}
% ---------------------------

% note that paths are relative to the working directory (the directory you build the LaTeX file from).
% Here, it's assumed that the working directory is one level higher than the directory that includes
% this file. In other words, it's the parent directory of `src`.

\linespread{1}

\begin{document}

    \thispagestyle{fancy}
    \fancyhf{}
    \lhead{2022 Δεκέμβριος}
    
    % --- Here, you enter the title and other relevant information ---
    \hspace{90ex}\includegraphics[width=0.4\textwidth]{../res/img/logo.png}\par\vspace{1ex} % Logo (ΕΚΠΑ) -- change it with appropriate logo, if necessary. 
 	
    {\Huge\centering ΑΝΥΣΜΑΤΑ, ΤΑΝΥΣΤΕΣ ΚΑΙ ΜΕΤΡΙΚΟΙ ΧΩΡΟΙ\par}\vspace{3ex} % General title
    {\Large\centering Μαμαλούκας Χριστόφορος-Μάριος\par}\vspace{2ex} % name of author
    {\centering Εθνικό και Καποδιστριακό Πανεπιστήμιο Αθηνών - Τμήμα Φυσικής\par}\vspace{4ex} % other info (here, date and tutor) 
    % ----------------- MAIN TEXT -----------------
    {\centering\textbf{ΠΕΡΙΛΗΨΗ}\par}\vspace{1ex}
    {\centering Το έγγραφο αυτό έχει ως στόχο να παρουσιάσει ορισμένες μαθηματικές έννοιες ανυσμάτων με συνεκτικό τρόπο. Εδώ, παρουσιάζονται
    	έννοιες που θεωρώ είναι σημαντικές να γνωρίζει κάποιος για να μελετήσει την Φυσική σύμφωνα με τον σύγχρονό της φορμαλισμό. Όσες προτάσεις για τις οποίες η απόδειξη
    	θεωρείται προφανής, θα αποκαλούνται «πορίσματα». Για τις προτάσεις με απόδειξη, οι οποίες θα αποκαλούνται «θεωρήματα», η τελευταία θα παρατίθεται
    	στο υπόμνημα.\par\vspace{3ex}} % Insert here the purpose of the project/the abstract.

    %--- The theory and the problem ---
    \section{\raggedright ΣΩΜΑΤΑ}

    \subfile{fields}
    % ---------------

    % --- The method ---
    \section{\raggedright ΑΝΥΣΜΑΤΑ}

    \subfile{vectors}
    % ------------------------------------------------

    % --- Data Analysis ---
    \section{\raggedright ΤΑΝΥΣΤΕΣ}

    \subfile{tensors}
    % ---------------------------

    % --- General Notes ---
    \section{\raggedright ΠΑΡΑΡΤΗΜΑ}

    \subfile{parartema}
    % ---------------------
    
    % --- Conclusion ---
    \section{\raggedright CONCLUSION}

    \subfile{conclusion1}
    % ---------------------
\end{document}

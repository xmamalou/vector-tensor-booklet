% Use this file to perform and showcase your data analysis

\documentclass[main.tex]{subfiles}

\begin{document}
	Αν και τα ανύσματα μας δίνουν μια ωραία γεωμετρική οπτική διαφόρων μαθηματικών δομών, δεν φαίνεται αρχικά να είναι πολύ ισχυρά ως έννοιες. Όμως, η Γραμμική Άλγεβρα ισχυρίζεται κάτι αλλόκοτο αλλά και πολύ όμορφο: Κάθε δομή η οποία εκφράζεται από γραμμικότητα μπορεί νto-
	\begin{definition}
		Έστω ανυσματικός χώρος $V$. Ονομάζουμε \textbf{δυϊκό χώρο} (dual space) του $V$ το σύνολο όλων των συναρτήσεων ${\{f:V\rightarrow\mathbb{F} : f \;\text{γραμμική}\}}$ και τον συμβολίζουμε $V^*$. Οι συναρτήσεις αυτές ονομάζονται \textbf{γραμμικά συναρτησοειδή} (linear functionals).
	\end{definition}

	Παραδείγματα γραμμικών συναρτησοειδών:
	\begin{itemize}
		\item χώρος $\mathbb{R}^3$: η συνάρτηση μέτρου $(\|\vec{r}\|)$, η συνάρτηση απόκλισης $(\vec\nabla\cdot\vec{r})$, η συνάρτηση συντεταγμένων $(x\;\text{ή}\;y\;\text{ή}\;z)$
		\item χώρος $P_5^{\mathbb{C}}$: το ολοκλήρωμα $(\int_{a}^{b}P(z)dz)$, συνάρτηση που δίνει τον συντελεστή μηδενικής τάξης του πολυωνύμου $(f(P) = P(0))$
	\end{itemize}

	\begin{corollary}
		Ο $V^*$ είναι ανυσματικός χώρος.
	\end{corollary}

	Αυτό είναι πολύ απλό να δειχθεί. Δείτε παρακάτω:
	\begin{align*}
		\text{Έστωσαν}&\;f,\;g\in V^*\\
		\text{Τότε}&\\
		f(\boldsymbol{u}) &+ g(\boldsymbol{u}) = (f + g)(\boldsymbol{u})\\
		af(\boldsymbol{u}) &= (af)(\boldsymbol{u})
	\end{align*}
	Τα ανωτέρω είναι προφανή και προκύπτουν από το πώς λειτουργούν οι συναρτήσεις. Η συνάρτηση $0:V\rightarrow\mathbb{F}$, με σχέση $0(\boldsymbol{u}) = e_0$, δρα ως το ουδέτερο στοιχείο της πρόσθεσης των στοιχείων αυτού του χώρου, και προφανώς κάθε συνάρτηση έχει αντίθετη. 
	
	Εφόσον, λοιπόν, τα στοιχεία του $V^*$ είναι ανύσματα, τότε μπορούμε να βρούμε για αυτά μία βάση $S^*$ και να τα γράψουμε ως γραμμικό συνδυασμό των ανυσμάτων αυτής της βάσης. Δηλαδή:
	\begin{align*}
		\boldsymbol{f}(\boldsymbol{u}) = \sum^{n}_{i=0}f_i\boldsymbol{e}^i(\boldsymbol{u})
	\end{align*}
	Όταν πλέον γράφουμε έναν συντελεστή με δείκτη κάτω (Π.χ. $u_i$), θα υπονοούμε ότι είναι συνιστώσα συναρτησοειδούς, σε αντιδιαστολή με τους συντελεστές με δείκτη πάνω (Π.χ. $u^i$), οι οποίοι θα αναφέρονται σε συνιστώσες ανυσμάτων.
	
	Όμως, εδώ έχουμε δύο προβλήματα:
	\begin{itemize}
		\item Δεν γνωρίζουμε το $n$, δηλαδή την διάσταση του δυϊκού χώρου, και χειρότερα
		\item Δεν γνωρίζουμε τις βάσεις $\boldsymbol{e}^i(\boldsymbol{u})$
	\end{itemize}
	Στα προβλήματα αυτά ευτυχώς υπάρχει λύση. Ας θυμηθούμε ότι τα ανύσματα του χώρου $V^*$ είναι γραμμικές συναρτήσεις, ως προς τα ανύσματα του $V$. Αυτό σημαίνει ότι αν έχουμε βάση $S$ του χώρου, μπορούμε να γράψουμε τον ανωτέρω συνδυασμό ως:
	\begin{align*}
		\boldsymbol{f}(\boldsymbol{u}) = \sum_{i=0}^{n}\sum_{j=0}^{\dim(V)}f_iu^j\boldsymbol{e}^i(\boldsymbol{e}_j) \; (1)
	\end{align*}
	Εφόσον η συνάρτηση $\boldsymbol{f}(\boldsymbol{u})$ μας δίνει βαθμωτό, και οι συντελεστές $f_iu^j$ είναι βαθμωτά, σημαίνει ότι μπορούμε να επιλέξουμε μια βάση $S^*$ τέτοια ώστε να ισχύει το εξής:
	\begin{align*}
		\boldsymbol{e}^i(\boldsymbol{e}_j) = \delta^i_{j} = \delta_{ij}
	\end{align*}
	Αυτές οι βάσεις ονομάζονται \textbf{συναρτήσεις συντεταγμένων}. Είναι πανομοιότυπες με αυτές που παρουσιάστηκαν στο εδάφιο (2.4.1), αφού η $\boldsymbol{e}^i(\boldsymbol{u})$ δίνει, προφανώς, αποτέλεσμα $u^i$.
	
	\begin{corollary}
		Έστω ανυσματικός χώρος $V$. Ο δυϊκός του χώρος $V^*$ έχει διάσταση ${\dim(V^*) = \dim(V)}$.
	\end{corollary}

	Ένα καταπληκτικό αποτέλεσμα από τα ανωτέρω είναι ότι μπορούμε να αναγράψουμε κάθε γραμμικό συναρτησοειδές ως απλό άθροισμα αριθμών. Συγκεκριμένα, απλοποιείται στην μορφή.
	\begin{align*}
		\boldsymbol{f}(\boldsymbol{u}) = \sum_{i=0}^{\dim(V)}f_iu^i \; (2)
	\end{align*}
	\begin{definition}
		Η ποσότητα $u^i$ ονομάζεται \textbf{ανταλλοίωτο} (contravariant) και η ποσότητα $f_i$ \textbf{συναλλοίωτο} (covariant) 
	\end{definition}

	Την ανωτέρω γραφή θα την απλοποιήσουμε, καθώς θα εμφανίζεται συχνά, στην \textbf{αθροιστική σύμβαση του Einstein}:
	\begin{definition}
		Ορίζουμε ως \textbf{αθροιστική σύμβαση του Einstein} (Einstein sum notation) την συντομογραφία:
		\begin{align*}
			a_ib^i \equiv \sum_{i\in I}a_ib^i,\; I\;\text{σύνολο δεικτών}
		\end{align*}
	\end{definition}
	Η ανωτέρω γραφή, λοιπόν, έχει νόημα όταν:
	\begin{itemize}
		\item To αντικείμενο με τον δείκτη \textbf{κάτω} είναι βάση ανύσματος και το αντικείμενο με τον δείκτη \textbf{πάνω} είναι ανταλλοίωτο. Η αθροιστική σύμβαση τότε αναφέρεται σε γραμμικό συνδυασμό ανύσματος.
		\item To αντικείμενο με τον δείκτη \textbf{κάτω} είναι συναλλοίωτο και το αντικείμενο με τον δείκτη \textbf{πάνω} είναι ανταλλοίωτο. Η αθροιστική σύμβαση τότε αναφέρεται στην τιμή ενός συναρτησοειδούς που δρα σε ένα άνυσμα.
	\end{itemize}

	Λογικό είναι η επιλογή βάσης που επιλέχθηκε να φανεί αυθαίρετη και ίσως ουρανοκατέβατη σε πολλούς. Αλλά υπάρχει απόλυτο μαθηματικό νόημα για αυτήν την κίνηση. Καθώς οι συντεταγμένες ενός ανύσματος $\boldsymbol{u}$ είναι... τεταγμένες, αυτό σημαίνει ότι δυνάμεθα να αναπαραστήσουμε το άνυσμα ως έναν πίνακα $\dim(V)\times 1$ με στοιχεία τις συντεταγμένες του:
	\begin{align*}
		\boldsymbol{u} = \begin{bmatrix}
			u^1 \\
			u^2 \\
			\ldots \\
			u^{\dim(V)}
		\end{bmatrix}
	\end{align*}
	Με έναν πίνακα μπορούμε να αναπαραστήσουμε και τα συναρτησοειδή, αφού είναι ανύσματα και αυτά. Το πρόβλημα είναι ότι δεν είναι πίνακες ${\dim(V)\times 1}$. Αλλά η σχέση (1) μας δίνει την μορφή του πίνακα. Η σχέση $\boldsymbol{f}(\boldsymbol{u})$ δίνει αριθμό ως αποτέλεσμα. Ποια πράξη μεταξύ δύο πινάκων γνωρίζουμε η οποία να δίνει αριθμό ως αποτέλεσμα; Η απάντηση είναι ο πολλαπλασιασμός ενός οριζόντιου πίνακα με έναν κάθετο. Ο κάθετος έχει ήδη ορισθεί να είναι ο $\boldsymbol{u}$ άρα έχουμε τον οριζόντιο πίνακα $\boldsymbol{f}$:
	\begin{align*}
		\boldsymbol{f}^T = \begin{bmatrix}
			f_1 & f_2 & \ldots & f_{\dim(V)}
		\end{bmatrix}
	\end{align*}
	και
	\begin{align*}
		\boldsymbol{f}(\boldsymbol{u}) = \boldsymbol{f}^T\cdot\boldsymbol{u} 
	\end{align*}
	απ' όπου προκύπτει ότι η επιλογή βάσης συναρτησοειδούς ως $\boldsymbol{e}^i(\boldsymbol{e}_j) = \delta_{ij}$ είναι δόκιμη επιλογή.
	
	Το ανωτέρω δεν αφορά μόνο διανύσματα του χώρου $\mathbb{R}^n$. Γίνεται υπενθύμιση ότι οι συντεταγμένες είναι αριθμοί σε οποιονδήποτε ανυσματικό χώρο, οπότε έχει νόημα να ορίσουμε πίνακα με αυτές για οποιονδήποτε χώρο. Για χώρους με ${\dim(V) = \aleph_0}$ (πεπερασμένους απειροδιάστατους), ο πίνακας είναι και αυτός απειροδιάστατος. Για χώρους με ${\dim(V) = \beth_1}$ (συνεχείς απειροδιάστατους), τα πράγματα αλλάζουν λίγο. Οι συναρτήσεις συντεταγμένων αλλάζουν στην κάτωθι μορφή:
	\begin{align*}
		\boldsymbol{e}(\boldsymbol{u};i) = \int^{+\infty}_{-\infty}u(k)\delta(i - k)dk = u(i)
	\end{align*}
	Άρα ο γραμμικός συνδυασμός ενός συναρτησοειδούς συνεχούς απειροδιάστατου χώρου γράφεται ως:
	\begin{align*}
		\boldsymbol{f}(\boldsymbol{u}) &= \int^{+\infty}_{-\infty}\tilde{f}(k)\boldsymbol{e}(\boldsymbol{u};k)dk \\
		&= \int^{+\infty}_{-\infty}\tilde{f}(k)\int^{+\infty}_{-\infty}u(k')\delta(k - k')dk'dk \\
		&= \int^{+\infty}_{-\infty}\tilde{f}(k)u(k)dk
	\end{align*}
	Δηλαδή έχουμε ακόμη την όμορφη αριθμητική σχέση που βρήκαμε με τους πεπερασμένους χώρους. Εδώ απλά το διακριτό άθροισμα γίνεται συνεχές. Για τα επόμενα, θα αγνοήσουμε, ως επί το πλείστον, τους συνεχείς απειροδιάστατους χώρους.
	
	Ένα ερώτημα που ίσως τεθεί είναι αν υπάρχει ένα συναρτησοειδές αντίστοιχο για κάθε άνυσμα του χώρου $V$. Η απάντηση είναι πως το ερώτημα αυτό είναι πολύ γενικό για να απαντηθεί. Αλλά, με την εισαγωγή του εσωτερικού γινομένου και κατ'επέκτασιν της νόρμας, μπορούμε να δώσουμε νόημα σε αυτό το ερώτημα.
	
	Η νόρμα $\|\cdot\|$ είναι προφανώς συναρτησοειδές και άρα μπορεί κάποιος να την γράψει στην μορφή
	\begin{align*}
		\|\boldsymbol{u}\|^2 \equiv u_iu^i
	\end{align*}
	Κατ'αντιστοιχίαν με την γραφή $\boldsymbol{u}^2$ της Ευκλείδειας Γεωμετρίας. Όμως τι είναι το συναλλοίωτο $u_i$; Την απάντηση την δίνει το εσωτερικό γινόμενο. Αν θυμηθούμε τα προηγούμενα, η νόρμα ορίζεται, ως προς το εσωτερικό γινόμενο, ως ${\|\boldsymbol{u}\|\ \equiv \langle\boldsymbol{u},\boldsymbol{u}\rangle}$.
	
	To εσωτερικό γινόμενο ανήκει σε έναν χώρο $\mathcal{T}^2(V)$ που ονομάζεται χώρος των διγραμμικών απεικονίσεων, το σύνολο, δηλαδή, ${\{\tau:V\times V \rightarrow \mathbb{F} | \;\tau \; \text{διγραμμική}\}}$, και ο οποίος είναι επίσης ανυσματικός χώρος, όπως απλά φαίνεται. Ως εκ τούτου, μπορούμε να βρούμε και για τα ανύσματα του χώρου αυτού μία βάση:
	\begin{align*}
		\boldsymbol{\tau}(\boldsymbol{u},\boldsymbol{v}) = \sum^{n}_{i=0}\tau_i\boldsymbol{e}^i(\boldsymbol{u},\boldsymbol{v})
	\end{align*}
	Και με πανομοιότυπα επιχειρήματα με τα συναρτησοειδή, καταλήγουμε στην μορφή:
	\begin{align*}
		\boldsymbol{\tau}(\boldsymbol{u},\boldsymbol{v}) = \sum^{n}_{i=0}\sum^{\dim(V)}_{j=0}\sum_{k=0}^{\dim(V)}\tau_iu^jv^k\boldsymbol{e}^i(\boldsymbol{e}_j,\boldsymbol{e}_k)
	\end{align*}
	Εδώ παρατηρούμε μία απόκλιση από την προηγούμενη διαδικασία. Το νέο $\boldsymbol{e}^i$ πλέον εξαρτάται και από το $\boldsymbol{e}_j$ και από το $\boldsymbol{e}_k$. Δηλαδή, στην πραγματικότητα, το i είναι δύο δείκτες, και μπορούμε να αναγράψουμε την βάση του χώρου των διγραμμικών απεικονίσεων ως το γινόμενο:
	\begin{align*}
		\boldsymbol{e}^{k\ell}(\boldsymbol{e}_i,\boldsymbol{e}_j) = \delta^k_i\delta^\ell_j = \delta_{ki}\delta_{\ell j}
	\end{align*}
	Και άρα, καταλήγουμε σε ένα ακόμη εκπληκτικό, αν και πλέον αναμενόμενο αποτέλεσμα: οι διγραμμικές απεικονίσεις μπορούν να αναγραφούν ως αριθμητικό άθροισμα. Δηλαδή:
	\begin{align*}
		\boldsymbol{\tau}(\boldsymbol{u},\boldsymbol{v}) = \tau_{ij}u^iv^j
	\end{align*}
	\begin{corollary}
		Η διάσταση του χώρου $\mathcal{T}^2(V)$ είναι $\dim(\mathcal{T}^2(V)) = (\dim(V))^2$
	\end{corollary}

	Προφανώς και το εσωτερικό γινόμενο αναγράφεται έτσι. Πλέον, θα συμβολίζουμε έτσι το εσωτερικό γινόμενο:
	\begin{align*}
		\langle\boldsymbol{u},\boldsymbol{v}\rangle = g_{ij}u^iv^j
	\end{align*}
	Αν απομονώσουμε το πρώτο κομμάτι της σχέσης αυτής, παρατηρούμε ότι εκφράζει συναρτησοειδές (άθροιση ως προς $i$ και ένας ελεύθερος δείκτης, ο $j$, ο οποίος είναι κάτω). Ως εκ τούτου, ορίζουμε την ακόλουθη πράξη:
	\begin{definition}
		Ονομάζουμε \textbf{καταβίβαση δείκτη} (lowering of the index) την πράξη:
		\begin{align*}
			u_i = g_{ji}u^j
		\end{align*}
		όπου $g_{ij}$ είναι η μετρική του χώρου $V$.
	\end{definition}
	Τώρα λοιπόν ξέρουμε πώς μπορεί να εκφραστεί η νόρμα με την απλή αλγεβρική μορφή:
	\begin{align*}
		\|\boldsymbol{u}\|^2 = \langle\boldsymbol{u},\boldsymbol{u}\rangle = g_{ij}u^iu^j = u_ju^j
	\end{align*}

	Οι διγραμμικές απεικονίσεις επεκτείνονται σε περισσότερες «διαστάσεις»
	\begin{definition}
		Έστω ανυσματικός χώρος $V$. Ονομάζουμε $k$-γραμμική απεικόνιση κάθε απεικόνιση της μορφής $\tau:\prod_{k}^{i=1}V\rightarrow \mathbb{F}$ (όπου $\prod_{k}^{i=1}V \equiv V\times V\times \overset{k \; \text{φορές}}{\ldots}$), η οποία ικανοποιεί τα εξής αξιώματα:
		\begin{itemize}
			\item $\forall i\leq k[\tau(\boldsymbol{u}_1, \boldsymbol{u}_2,\ldots,(\boldsymbol{v}_{1} + \boldsymbol{v}_{2})_i,\ldots,\boldsymbol{u}_k) = \tau(\boldsymbol{u}_1, \boldsymbol{u}_2,\ldots,\boldsymbol{v}_{1},\ldots,\boldsymbol{u}_k) + \tau(\boldsymbol{u}_1, \boldsymbol{u}_2,\ldots,\boldsymbol{v}_{2},\ldots,\boldsymbol{u}_k)]$
			\item $\forall i\leq k[\tau(\boldsymbol{u}_1, \boldsymbol{u}_2,\ldots,a\boldsymbol{u}_i,\ldots,\boldsymbol{u}_k) = a\tau(\boldsymbol{u}_1, \boldsymbol{u}_2,\ldots,\boldsymbol{u}_i,\ldots,\boldsymbol{u}_k)]$
		\end{itemize}
		δηλαδή την απεικόνιση η οποία είναι γραμμική ως προς κάθε όρισμά της. Τον χώρο όλων αυτών των απεικονίσεων των συμβολίζουμε $\mathcal{T}^k(V)$
	\end{definition}
	Και πάλι, ο χώρος των $k$-γραμμικών απεικονίσεων είναι ανυσματικός χώρος.
	
	\begin{corollary}
		Η διάσταση του χώρου $\mathcal{T}^k(V)$ είναι $\dim(\mathcal{T}^k(V)) = (\dim(V))^k$
	\end{corollary}

	Για τις $k$-γραμμικές απεικονίσεις, η γραφή τους ως αριθμητικό άθροισμα γενικεύεται εύκολα από την μορφή της γραφής για την διγραμμική απεικόνιση. Συγκεκριμένα, έχουμε:
	\begin{align*}
		\boldsymbol{\tau}(\boldsymbol{a},\ldots,\boldsymbol{z}) = \tau_{i_1i_2\ldots i_k}a^{i_1}b^{i_2}\ldots z^{i_k}
	\end{align*}
	
	\begin{definition}
		Το αντικείμενο $\tau_{i_1i_2\ldots i_k}$ ονομάζεται \textbf{τανυστής} (tensor) τάξης $(0,k)$ ή, αλλιώς, \textbf{συναλλοίωτος τανυστής} (cotensor) τάξης $k$
	\end{definition}

	\begin{corollary}
		Τα συναλλοίωτα είναι τανυστές τάξης $(0,1)$
	\end{corollary}
	
	Υπενθύμιση ότι έχει μεγάλη σημασία το αν ο δείκτης (ή οι δείκτες) είναι πάνω ή κάτω, καθώς δηλώνει την προέλευση του αντικειμένου με τον δείκτη. Σε ύστερο κομμάτι αυτού του φυλλαδίου, θα δούμε ότι υπάρχουν και αναμίξεις των θέσεων των δεικτών. Δεν θα είναι απαραίτητο να είναι όλοι πάνω ή κάτω. Επίσης, ο ορισμός (3.0.6) δεν είναι επίσημος ορισμός του τανυστή. Αυτός θα δωθεί αργότερα.

	Κάτι που πρέπει επίσης να θυμηθούμε σχετικά με τα ανωτέρω είναι το εξής:
	Τα ανταλλοίωτα και οι τανυστές είναι ποσότητες πλήρως εξαρτημένες από τον ανυσματικό χώρο και την βάση του. \textbf{Δεν έχει νόημα} να μιλάμε για τα προαναφερθέντα αν δεν έχουμε \textbf{ορίσει} τον χώρο στον οποίο αναφέρονται. Εκτός αυτού, καθώς κρύβουν μέσα τους συντεταγμένες, αυτό σημαίνει ότι αλλάζουν αν αλλάξει η βάση. Οπότε, τώρα τίθεται το ερώτημα; Πώς μετασχηματίζονται τα ανταλλοίωτα και οι τανυστές, όταν μετασχηματιστεί μια βάση;
	
	Από το θεώρημα $(2.1.2)$ γνωρίζουμε ότι, υπό προϋποθέσεις, αν έχουμε βάση $S$ ενός ανυσματικού χώρου, τότε το σύνολο ${S' = \{\boldsymbol{e}_i + \boldsymbol{u}_i\}}$, με ${\boldsymbol{e}_i \in S}$, είναι επίσης βάση του χώρου. Μπορούμε τις βάσεις του $S'$ να τις εκφράσουμε ως γραμμικό συνδυασμό των βάσεων του $S$:
	\begin{align*}
		\boldsymbol{e}'_i = \sum^{dim(V)}_{j=1}C^j_i\boldsymbol{e}_j
	\end{align*}
	όπου $C^i_j \in \mathbb{F}$ ένας αριθμός. Για ένα άνυσμα $\boldsymbol{u}$, πώς μετασχηματίζονται οι συντεταγμένες του από την βάση $S$ στην βάση $S'$; Το γράφουμε ως γραμμικό συνδυασμό και των δύο βάσεων:
	\begin{align*}
		\sum_{i=1}^{dim(V)}u'^i\boldsymbol{e}'_i &= \sum_{i=1}^{dim(V)}u^i\boldsymbol{e}_i \Leftrightarrow \\
		\sum_{i=1}^{dim(V)}u'^i\sum^{dim(V)}_{j=1}C^j_i\boldsymbol{e}_j &= \sum_{i=1}^{dim(V)}u^i\boldsymbol{e}_i \Leftrightarrow \\
		\sum_{i=1}^{dim(V)}\sum^{dim(V)}_{j=1}C^j_iu'^i\boldsymbol{e}_j &= \sum_{i=1}^{dim(V)}u^i\boldsymbol{e}_i \Leftrightarrow \\
		\sum_{j=1}^{dim(V)}\sum^{dim(V)}_{i=1}C^j_iu'^i\boldsymbol{e}_i &= \sum_{i=1}^{dim(V)}u^i\boldsymbol{e}_i \Leftrightarrow \\
		C^i_ju'^j &= u^i
	\end{align*}
	Παρατηρούμε ότι το αντικείμενο που μετασχηματίζει τις βάσεις από την $S$ στην $S'$ είναι το αντικείμενο που -παραδόξως- μετασχηματίζει τα ανταλλοίωτα από την βάση $S'$ στην βάση $S$. 
	\begin{definition}
		Τα αντικείμενα $T^i_j$ ονομάζονται \textbf{γραμμικοί μετασχηματισμοί} (linear transformations).
	\end{definition}
	Φυσικά, θέλουμε να ξέρουμε πώς να μετασχηματίσουμε τα ανταλλοίωτα βάσης $S$ σε $S'$. Μία απρόσεκτη σκέψη που μπορεί κάποιος να κάνει είναι να διαιρέσουμε με την ποσότητα $C^i_j$. Όμως, αυτό είναι μεγάλο ατόπημα, καθώς οι εκφράσεις της μορφής $C^i_ju^j$ κρύβουν μέσα τους άθροισμα (ένας επαναλαμβανόμενος δείκτης πάνω και κάτω).
	Το πρόβλημά μας πάλι θα λυθεί με την βοήθεια των πινάκων.
	
	Οι αριθμοί $C^i_j$ μπορούν να γραφούν με την μορφή πίνακα ως κάτωθι:
	\begin{align*}
		\boldsymbol{C} = \begin{bmatrix}
			C^1_1 & C^1_2 & \ldots & C^1_{\dim(V)} \\
			C^2_1 & \ddots & & \vdots\\
			\vdots & & \ddots & \vdots \\
			C^{\dim(V)}_1 & C^{\dim(V)}_2 & \ldots & C^{\dim(V)}_{\dim(V)}
		\end{bmatrix}
	\end{align*}
	Άρα, η σχέση στην οποία καταλήξαμε πριν αναγράφεται ως εξής:
	\begin{align*}
		\boldsymbol{C}\cdot\boldsymbol{u'} = \boldsymbol{u}
	\end{align*}
	Είναι εύκολο να δειχθεί ότι ο πίνακας $\boldsymbol{C}$ είναι ο πίνακας $U_{ij}$ στον οποίο αναφέρεται το θεώρημα $(2.1.2)$. Ως εκ τούτου, ξέρουμε ότι η ορίζουσά του είναι μη μηδενική και, άρα, έχει αντίστροφο πίνακα:
	\begin{align*}
		\boldsymbol{C}\cdot\boldsymbol{u}' &= \boldsymbol{u} \Leftrightarrow\\
		\boldsymbol{C}^{-1}\cdot\boldsymbol{C}\cdot\boldsymbol{u}' &= \boldsymbol{C}^{-1}\cdot\boldsymbol{u} \Leftrightarrow\\
		\boldsymbol{u}' &= \boldsymbol{C}^{-1}\cdot\boldsymbol{u}'
	\end{align*}
	Με άλλα λόγια, ο πίνακας που μετασχηματίζει τις συντεταγμένες (τα ανταλλοίωτα) από την βάση $S$ στην βάση $S'$ είναι ο αντίστροφος του πίνακα που μετασχηματίζει τα ανύσματα της βάσης $S$ στην βάση $S'$. 
	
	Τώρα πρέπει να βρούμε πώς να μετασχηματίσουμε τους τανυστές από την βάση $S$ στην βάση $S'$. Έστω μετασχηματισμός ανταλλοιώτων $T^i_j$ μεταξύ των δύο βάσεων. Ισχύει:
	\begin{align*}
		\tau'_{i_1i_2\ldots i_k}a'^{i_1}b'^{i_2}\ldots z'^{i_k} &= \tau_{i_1i_2\ldots i_k}a^{i_1}b^{i_2}\ldots z^{i_k} \Leftrightarrow \\
		\tau'_{i_1i_2\ldots i_k}T^{i_1}_{j_1}a^{j_1}T^{i_2}_{j_2}b^{j_2}\ldots T^{i_k}_{j_k}z'^{j_k} &=  \tau_{i_1i_2\ldots i_k}a^{i_1}b^{i_2}\ldots z^{i_k} \Leftrightarrow \\
		T^{i_1}_{j_1}T^{i_2}_{j_2}\ldots T^{i_k}_{j_k}\tau'_{i_1i_2\ldots i_k}a^{j_1}b^{j_2}\ldots z^{j_k} &= \tau_{i_1i_2\ldots i_k}a^{i_1}b^{i_2}\ldots z^{i_k} \overset{\text{αντίστροφος μετ/σμός}}{\Leftrightarrow}\\
		(T^{-1})^{i_1}_{j_1}(T^{-1})^{i_2}_{j_2}\ldots(T^{-1})^{i_k}_{j_k}T^{i_1}_{j_1}T^{i_2}_{j_2}\ldots T^{i_k}_{j_k}\tau'_{i_1i_2\ldots i_k}a^{j_1}b^{j_2}\ldots z^{j_k} &= (T^{-1})^{j_1}_{i_1}(T^{-1})^{j_2}_{i_2}\ldots(T^{-1})^{j_k}_{i_k}\tau_{i_1i_2\ldots i_k}a^{i_1}b^{i_2}\ldots z^{i_k}\Leftrightarrow\\
		\tau'_{i_1i_2\ldots i_k}a^{j_1}b^{j_2}\ldots z^{j_k} &= (T^{-1})^{j_1}_{i_1}(T^{-1})^{j_2}_{i_2}\ldots(T^{-1})^{j_k}_{i_k}\tau_{j_1j_2\ldots j_k}a^{i_1}b^{i_2}\ldots z^{i_k}
	\end{align*}
	Ως εκ τούτου, καταλήγουμε στο κάτωθι:
	\begin{theorem}
		Ένας συναλλοίωτος τανυστής τάξης $k$, που προκύπτει από ανυσματικό χώρο $V$, μετασχηματίζεται από βάση $S$ σε βάση $S'$ ως κάτωθι:
		\begin{align*}
			\tau'_{i_1i_2\ldots i_k} &= (T^{-1})^{j_1}_{i_1}(T^{-1})^{j_2}_{i_2}\ldots(T^{-1})^{j_k}_{i_k}\tau_{j_1j_2\ldots j_k}
		\end{align*}
		όπου $T^{-1}$ συμβολίζει τον αντίστροφο γραμμικό μετασχηματισμό ανταλλοίωτων από την βάση $S$ στην βάση $S'$.
	\end{theorem}

	\subsection{Γραμμικοί τελεστές}
	
	\begin{definition}
		Ονομάζουμε \textbf{γραμμικούς τελεστές} (linear operators), ενός ανυσματικού χώρου $V$, συναρτήσεις ${\hat{F}:V\rightarrow V}$ για τις οποίες ισχύει:
		\begin{itemize}
			\item $\hat{F}(\boldsymbol{u} + \boldsymbol{v}) = \hat{F}(\boldsymbol{u}) + \hat{F}(\boldsymbol{v})$ 
			\item $\hat{F}(a\boldsymbol{u}) = a\hat{F}(\boldsymbol{u})$
		\end{itemize}
		δηλαδή είναι γραμμικές ως προς τα ανύσματα του χώρου $V$.
	\end{definition}

	Το συνηθέστερο παράδειγμα γραμμικού τελεστή είναι η παράγωγος. Πράγματι, παρατηρούμε ότι, π.χ., στον χώρο $\mathbb{R}^3$
	\begin{align*}
		\frac{\partial\vec{f}}{\partial t} = \vec{f}_t
	\end{align*}
	Δηλαδή ο τελεστής $\frac{\partial}{\partial t}$ παίρνει ένα διανυσματικό πεδίο και δίνει ένα άλλο διανυσματικό πεδίο.
	
	Επίσης συχνό παράδειγμα γραμμικού τελεστή, και το οποίο ήδη είδαμε, είναι οι γραμμικοί μετασχηματισμοί.
	
	Παραδείγματα λίγο πιο «κρυμμένων» γραμμικών τελεστών είναι η ροπή αδράνειας $I$ που εμφανίζεται στην μηχανική των στερεών και ο τανυστής ροής $\Pi$ της Ρευστομηχανικής.
	
	\begin{corollary}
		Ο χώρος των γραμμικών τελεστών $\mathcal{L}^1(V)$ είναι ανυσματικός χώρος.
	\end{corollary}
	
	Ως εκ τούτου, όπως και τα υπόλοιπα αντικείμενα που έχουμε δει ως τώρα, μπορεί να γραφεί ως γραμμικός συνδυασμός:
	\begin{align*}
		\hat{F}(\boldsymbol{u}) &= \sum_{i=1}^{n}F^i\hat{e}_i(\boldsymbol{u}) =\\
		&= \sum_{i=1}^{n}\sum_{j=1}^{\dim(V)}F^iu^j\hat{e}_i(\boldsymbol{e}_j)
 	\end{align*}
 	Τώρα πρέπει να βρούμε έκφραση για την $\hat{e}_i(\boldsymbol{e}_j)$ Γνωρίζουμε ότι η έκφραση αυτή πρέπει να δίνει ως απάντηση άνυσμα. Θα χρησιμοποιήσουμε επιχειρήματα παρόμοια με την βάση του δυϊκού χώρου. Όπως και πριν, η απάντηση θα δοθεί από τους πίνακες. Ήδη έχουμε θέσει τις συντεταγμένες του ανύσματος $\boldsymbol{u}$ να σχηματίζουν έναν κάθετο πίνακα. Επίσης, γνωρίζουμε ότι ο πολλαπλασιασμός ενός τετραγωνικού πίνακα με ένα κάθετο πίνακα δίνει πάλι κάθετο πίνακα. Έτσι, καταλαβαίνουμε ότι το $i$ είναι στην πραγματικότητα δύο δείκτες και μια κατάλληλη επιλογή βάσης του χώρου των γραμμικών τελεστών θα ήταν:
 	\begin{align*}
 		\hat{e}_k^\ell(\boldsymbol{e}_j) = \delta_{\ell j}\boldsymbol{e}_k
 	\end{align*}
	άρα, έχουμε:
	\begin{align*}
		\hat{F}(\boldsymbol{u}) &= \sum_{k=1}^{\dim(V)}\sum_{\ell=0}^{\dim(V)}\sum_{j=0}^{\dim(V)}F^k_\ell u^j\delta_{\ell j}\boldsymbol{e}_k =\\
		&= \sum_{k=1}^{\dim(V)}\sum_{\ell=0}^{\dim(V)}F^k_\ell u^\ell\boldsymbol{e}_k
	\end{align*}
	Χρησιμοποιώντας την αθροιστική σύμβαση του Einstein, έχουμε ότι η $k$ συνιστώσα του ανύσματος $\hat{F}\boldsymbol{u}$ είναι:
	\begin{align*}
		F^k_\ell u^\ell
	\end{align*}
	
	Προφανώς, εφόσον ο δυϊκός χώρος $V^*$ του $V$ είναι επίσης ανυσματικός χώρος, έχει εξίσου νόημα να μιλήσουμε για τους γραμμικούς τελεστές ${\tilde{F}:V^*\rightarrow V^*}$, δηλαδή αυτούς που δρουν πάνω σε συναρτησοειδή. Ένα παράδειγμα είναι, πάλι, η παράγωγος.
	
	Ακολουθώντας την ίδια ακριβώς διαδικασία με τους τελεστές του χώρου $V$, καταλήγουμε ότι $k$ συνιστώσα ενός συναρτησοειδούς $\tilde{F}\boldsymbol{f}$ είναι:
	\begin{align*}
		F_k^\ell f_\ell
	\end{align*}

	Και οι δύο έχουν ακριβώς την ίδια τάξη αλλά και την ίδια διάσταση
	\begin{corollary}
		Έστωσαν ανυσματικός χώρος $V$ και ο δυϊκός του χώρος $V^*$. Οι χώροι των γραμμικών τους τελεστών $\mathcal{L}^1(V)$ και $\mathcal{L}^1(V^*)$ έχουν διάσταση $\dim(\mathcal{L}^1(V)) = \dim(\mathcal{L}^1(V^*)) = (\dim(V))^2 = (\dim(V^*))^2$
	\end{corollary}

	Όπως και με τις γραμμικές απεικονίσεις, μπορούμε να γενικεύσουμε τους γραμμικούς τελεστές:
	\begin{definition}
		Ονομάζουμε \textbf{$k$-γραμμικό τελεστή} (k-linear operator), κάθε συνάρτηση $f:\prod_{i=1}^{k}V\rightarrow V$ ή $f:\prod_{i=1}^{k}V^*\rightarrow V^*$, η οποία είναι γραμμική ως προς κάθε της όρισμα.
	\end{definition}

	Για τους πολυγραμμικούς τελεστές των ανταλλοιώτων και των συναλλοιώτων, καταλήγουμε, πάλι μέσω γραμμικών συνδυασμών, στα αθροίσματα:
	\begin{align*}
		&F^\mu_{i_1i_2\ldots i_k}u^{i_1}u^{i_2}\ldots u^{i_k} \; \text{πολυγραμμικός τελεστής ανταλλοιώτων} \\
		&F_\mu^{i_1i_2\ldots i_k}u_{i_1}u_{i_2}\ldots u_{i_k} \; \text{πολυγραμμικός τελεστής συναλλοιώτων} \\
	\end{align*}
	
	Τα αντικείμενα $F^\mu_{i_1i_2\ldots i_k}$ και $F_\mu^{i_1i_2\ldots i_k}$ είναι τανυστές τάξης $(1,k)$ και $(k,1)$ αντιστοίχως και ανήκουν στους χώρους $\mathcal{L}^k(V)$ και $\mathcal{L}^{k}(V^*)$ αντιστοίχως.
	
	\begin{corollary}
		Έστωσαν ανυσματικός χώρος $V$ και ο δυϊκός του χώρος $V^*$. Οι χώροι των πολυγραμμικών τους τελεστών $\mathcal{L}^k(V)$ και $\mathcal{L}^k(V^*)$ έχουν διάσταση $\dim(\mathcal{L}^k(V)) = \dim(\mathcal{L}^k(V^*)) = (\dim(V))^{k+1} = (\dim(V^*))^{k+1}$
	\end{corollary}
	
	Το επόμενο βήμα που ενδεχομένως κάποιος να σκεφτόταν είναι τι θα πάρουμε αν συνδυάσουμε ένα συναρτησοειδές πάνω στο οποίο έχει δράσει ένας τελεστής (τάξης $(k,1)$) του και ένα άνυσμα στο οποίο επίσης έχει δράσει τελεστής (τάξης $(1,\ell)$). Θα έχουμε λοιπόν:
	\begin{align*}
		&\tilde{F}\boldsymbol{f}(\hat{F}\boldsymbol{u}) = \\
		&F_\mu^{i_1i_2\ldots i_k}f_{i_1}f_{i_2}\ldots f_{i_k}\boldsymbol{e}^\mu(F^\nu_{j_1j_2\ldots j_\ell}u^{j_1}u^{j_2}\ldots u^{j_\ell}\boldsymbol{e}_\nu) =\\
		&F_\mu^{i_1i_2\ldots i_k}f_{i_1}f_{i_2}\ldots f_{i_k}F^\nu_{j_1j_2\ldots j_\ell}u^{j_1}u^{j_2}\ldots u^{j_\ell}\boldsymbol{e}^\mu(\boldsymbol{e}_\nu)
	\end{align*}
	Όμως, πριν, ορίσαμε την βάση $\boldsymbol{e}^\mu(\boldsymbol{e}_\nu)$ ως $\delta_{\mu\nu}$. Άρα, καταλήγουμε στο άθροισμα:
	\begin{align*}
		F_\mu^{i_1i_2\ldots i_k}F^\mu_{j_1j_2\ldots j_\ell}f_{i_1}f_{i_2}\ldots f_{i_k}u^{j_1}u^{j_2}\ldots u^{j_\ell}
	\end{align*}
	Επειδή οι δείκτες $\mu$ αθροίζονται, μπορούμε να συμπτήξουμε το πρώτο μέρος στο εξής αντικείμενο:
	\begin{align*}
		F^{i_1i_2\ldots i_k}_{j_1j_2\ldots j_\ell}
	\end{align*}

	Αυτό το αντικείμενο ονομάζεται \textbf{μεικτός τανυστής τάξης $(k,l)$}. Παρατηρούμε ότι η πράξη $\tilde{F}\boldsymbol{f}(\hat{F}\boldsymbol{u})$ καταλήγει σε μια μορφή που θυμίζει αυτήν των συναλλοίωτων τανυστών. Πράγματι, αυτή είναι η σχέση μίας πολυγραμμικής απεικόνισης, μόνον που αυτήν την φορά έχουμε ανάμειξη συναλλοιώτων και ανταλλοιώτων ποσοτήτων, εξού και η ονομασία «μεικτός». Με άλλα λόγια, καταλήγουμε στο συμπέρασμα:
	\begin{definition}
		Έστωσαν ανυσματικός χώρος $V$ και ο δυϊκός του χώρος $V^*$. Ορίζουμε ως χώρο $\mathcal{T}^k_\ell(V)$ των πολυγραμμικών απεικονίσεων του χώρου $V$ το σύνολο των συναρτήσεων ${\{\tau:\prod_{i=0}^{k}V^*\prod_{j=0}^{\ell}V \rightarrow \mathbb{F} : \tau \; \text{γραμμικές ως προς κάθε όρισμα}\}}$. Οι συναρτήσεις που ανήκουν στον χώρο $\mathcal{T}^k_\ell(V)$ ονομάζονται \textbf{μεικτοί τανυστές τάξης} (mixed tensors of rank) $(k,l)$ 
	\end{definition}
	(Για λόγους φορμαλισμού, ορίζουμε $\prod_{i=0}^{0}V^*\prod_{j=0}^{\ell}V = \prod_{j=0}^{\ell}V$, $\prod_{i=0}^{k}V^*\prod_{j=0}^{0}V = \prod_{i=0}^{k}V^*$ και $\prod_{i=0}^{0}V^*\prod_{j=0}^{0}V = \mathbb{F}$)

	Δεν θα έπρεπε να εκπλήξει κανέναν το γεγονός ότι ο χώρος των μεικτών τανυστών είναι επίσης ανυσματικός χώρος. Ο χώρος αυτός έχει διάσταση $\dim(\mathcal{T}^k_\ell(V)) = [\dim(V)]^{k+l}$.
	
	Έχοντας πλέον ορίσει συναλλοίωτους και μεικτούς τανυστές, φυσικό επόμενο είναι ο ορισμός των ανταλλοίωτων τανυστών:
	\begin{definition}
		Ονομάζουμε \textbf{ανταλλοίωτο τανυστή} (contravariant tensor) έναν τανυστή τάξης $(k,0)$, δηλαδή μία πολυγραμμική απεικόνιση $\tau:\prod_{i=0}^kV^*\rightarrow \mathbb{F}$
	\end{definition}
	Κάποιος μπορεί να διαπιστώσει ότι οι ανταλλοίωτοι τανυστές προκύπτουν από πράξεις της μορφής $\tilde{F}\boldsymbol{f}(\boldsymbol{u})$, δηλαδή από την δράση ενός τελεστή συναλλοίωτου σε αυτό. Βέβαια, είναι πιθανό να γίνει η συσχέτιση ότι τότε οι πολυγραμμικοί τελεστές $\mathcal{L}^k(V^*)$ είναι επίσης ανταλλοίωτοι τανυστές. Αυτό είναι \textbf{ΛΑΘΟΣ}. Αρχικά, ένας τανυστής πρέπει να είναι απεικόνιση \textbf{στο αλγεβρικό σώμα στο οποίο ορίζονται οι ανυσματικοί χώροι με τους οποίους ασχολούμαστε}. Οι τελεστές, αντ' αυτού, δεν μας στέλνουν στο αλγεβρικό σώμα, μας στέλνουν σε έναν γενικό ανυσματικό χώρο. Φορμαλιστικά μιλώντας, ο τελεστής αυτός καθ'αυτός του συναλλοίωτου ήταν «τυφλός» στο όρισμα του. Με άλλα λόγια, δεν ήξερε αν αυτό που μετασχηματίζεται είναι ανταλλοίωτο ή συναλλοίωτο άνυσμα. Το γεγονός ότι γράψαμε τους δείκτες πάνω ή κάτω είναι καθαρά για λόγους συνέπειας, καθώς δεν υπάρχει καμμιά ποιοτική διαφορά στην συμπεριφορά μεταξύ των τελεστών $\mathcal{L}^k(V)$ και των τελεστών $\mathcal{L}^k(V^*)$.
	
	\begin{corollary}
		Το ανταλλοίωτο είναι τανυστής $(1,0)$ τάξης.
	\end{corollary}

	Αρκετά πριν, είχαμε ορίσει την καταβίβαση δείκτη ως την πράξη $u_i = g_{ij}u^j$, όπου $g_{ij}$ η μετρική του ανυσματικού χώρου $V$. Προφανώς, θα ήταν χρήσιμο αν μπορούσαμε να κάνουμε το αντίστροφο, δηλαδή να βρούμε μία πράξη $u_i \rightarrow u^i$.
	
	Εφόσον θέλουμε να λάβουμε ένα ανταλλοίωτο, η πράξη αυτή θα είναι της μορφής
	\begin{align*}
		u^i = g^{ij}u_j
	\end{align*}
	Το ερώτημα είναι λοιπόν «τί είναι το $g^{ij}$»;
	\begin{align*}
		u^i &= g^{ij}u_j \Leftrightarrow\\
		u^i &= g^{ij}g_{jk}u_k
	\end{align*}
	Από αυτό, λοιπόν, παρατηρούμε ότι πρέπει να ισχύει η κάτωθι σχέση μεταξύ των ποσοτήτων $g^{ij}$ και $g_{ij}$
	\begin{align*}
		g^{ij}g_{jk} = \delta^i_k
	\end{align*}
	Παρατηρούμε ότι, \textbf{αριθμητικά}, ο ανταλλοίωτος τανυστής $g^{ij}$ είναι ίσος με τον αντίστροφο πίνακα της μετρικής.
	
	\begin{definition}
		Ονομάζουμε \textbf{αναβίβαση δείκτη} (raising of the index) την πράξη:
		\begin{align*}
			u^i = g^{ij}u_j
		\end{align*}
		όπου $g^{ij}$ είναι ένας ανταλλοίωτος τανυστής τάξης 2, ο οποίος σχετίζεται με την μετρική $g_{ij}$ του χώρου $V$ ως εξής:
		\begin{align*}
			g^{ij}g_{jk} = \delta^i_k = \delta_{ik}
		\end{align*} 
	\end{definition} 
	
	\subsection{Αναμειγνύοντας ανυσματικούς χώρους - Το τανυστικό γινόμενο}
	Τώρα που έχουμε μιλήσει για νέους χώρους που εμφανίζονται «φυσικά» από έναν χώρο $V$, θα ήταν βολικό αν μπορούσαμε να συνθέσουμε διαφορετικούς χώρους μεταξύ τους.
	
	\begin{definition}
		Έστωσαν ανυσματικοί χώροι $V$ και $W$ πεπερασμένης διάστασης, με βάσεις $S_V$ και $S_W$.
		
		Ονομάζουμε \textbf{ευθύ άθροισμα} (direct sum) τον ανυσματικό χώρο ${V\oplus W = V\times W}$ με ανύσματα τα οποία εκφράζονται ως κάτωθι:
		\begin{align*}
			(\boldsymbol{v}\oplus\boldsymbol{w}) = (\boldsymbol{u}, \boldsymbol{w})
		\end{align*}
		όπου ${\boldsymbol{u}\in V}$ και ${\boldsymbol{w}\in W}$ και με την πρόσθεση και βαθμωτό πολλαπλασιασμό να ορίζονται ως:
		\begin{align*}
			(\boldsymbol{u}_1, \boldsymbol{w}_1) + (\boldsymbol{u}_2, \boldsymbol{w}_2) &= (\boldsymbol{u}_1 + \boldsymbol{u}_2, \boldsymbol{w}_1 + \boldsymbol{w}_2)\\
			a(\boldsymbol{u}, \boldsymbol{w}) &= (a\boldsymbol{u}, a\boldsymbol{w})
		\end{align*}
	\end{definition}

	Γιατί η ονομασία ευθύ άθροισμα; Η ονομασία οφείλεται στο πώς λειτουργούν η πρόσθεση και ο βαθμωτός πολλαπλασιασμός. Συγκεκριμένα, δρουν πάνω στο $(\boldsymbol{u},\boldsymbol{w})$ σαν να αθροίζονταν τα στοιχεία του μεταξύ τους.
	
	Έχοντας ορίσει μία πράξη που μοιάζει με άθροισμα, θέλουμε τώρα να ορίσουμε μια πράξη η οποία μοιάζει με γινόμενο. Αυτή η πράξη είναι το τανυστικό γινόμενο.
	\begin{definition}
		Ορίζουμε ως \textbf{τανυστικό γινόμενο} (tensor product) δύο ανυσματικών χώρων $V$ και $W$ τον ανυσματικό χώρο $V\otimes W$, ο οποίος είναι το σύνολο αφίξεως της πράξης ${\otimes:V\times W \rightarrow V\otimes W}$, η οποία ικανοποιεί τις κάτωθι συνθήκες:
		\begin{itemize}
			\item Είναι διγραμμική ως προς τα ορίσματά της
			\item Για κάθε \textbf{διγραμμική} απεικόνιση ${\tau : V\times W \rightarrow Z}$ υπάρχει μοναδική \textbf{γραμμική} απεικόνιση ${\tilde{\tau}:V\otimes W \rightarrow Z}$ για την οποία να ισχύει το εξής: $\forall \boldsymbol{u} \in V, \boldsymbol{w} \in W[\tau(\boldsymbol{u},\boldsymbol{w}) = \tilde{\tau}(\otimes(\boldsymbol{u},\boldsymbol{w}))]$
 		\end{itemize}
 		Θα συμβολίζουμε την πράξη $\otimes(\boldsymbol{u},\boldsymbol{w})$ ως $(\boldsymbol{u}\otimes\boldsymbol{w})$
 	\end{definition}
 	Γιατί συμπεριφέρεται αυτή η πράξη ως γινόμενο; Σύμφωνα με τον ορισμό του, οι πράξεις της ανυσματικής πρόσθεσης και του βαθμωτού πολλαπλασιασμού είναι οι κάτωθι:
 	\begin{align*}
 		(\boldsymbol{u}_1\otimes\boldsymbol{w}) + (\boldsymbol{u}_2\otimes\boldsymbol{w}) &= [(\boldsymbol{u}_1 + \boldsymbol{u}_2)\otimes\boldsymbol{w}] \\
 		(\boldsymbol{u}\otimes\boldsymbol{w}_1) + (\boldsymbol{u}\otimes\boldsymbol{w}_2) &= [\boldsymbol{u}\otimes(\boldsymbol{w}_1 + \boldsymbol{w}_2)] \\
 		a(\boldsymbol{u}\otimes\boldsymbol{w}) = [(a\boldsymbol{u})\otimes\boldsymbol{w}] &= [\boldsymbol{u}\otimes(a\boldsymbol{w})]
 	\end{align*}
 	Αυτά θυμίζουν αρκετά την επιμεριστική και την προσεταιριστική ιδιότητα του πολλαπλασιασμού, εξού και η ονομασία.
\end{document}

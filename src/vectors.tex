% Use this file to describe the manner in which the experiment will be executed

\documentclass[main.tex]{subfiles}

\begin{document}
	Όπως και με τα σώματα, τα ανύσματα ορίζονται μέσω ενός συνόλου.
	
	\begin{definition}
		Ένας διανυσματικός χώρος $V$ (vector space) επί αλγεβρικού σώματος $\mathbb{F}$ ονομάζεται κάθε σύνολο του οποίου στοιχεία είναι \textbf{δομές} κατασκευασμένες με στοιχεία του $\mathbb{F}$ και έχουν δύο πράξεις, την \textbf{ανυσματική πρόσθεση} (vector addition) και τον \textbf{βαθμωτό πολλαπλασιασμό} (scalar multiplication), οι οποίες ικανοποιούν τα κάτωθι αξιώματα:
		\begin{enumerate}
			\item Ανυσματική Πρόσθεση:
			\begin{itemize}
				\item $\forall\boldsymbol{v},\boldsymbol{u}\in V\exists!\boldsymbol{w}\in V(\boldsymbol{v}+\boldsymbol{u} = \boldsymbol{w})$ (Κλειστότητα)
				\item $\forall\boldsymbol{v},\boldsymbol{u}\in V(\boldsymbol{v} + \boldsymbol{u} = \boldsymbol{u} + \boldsymbol{v})$ (Μεταθετικότητα)
				\item $\forall\boldsymbol{v},\boldsymbol{u},\boldsymbol{w}\in V[(\boldsymbol{v} + \boldsymbol{u}) + \boldsymbol{w} = \boldsymbol{v} + (\boldsymbol{u} + \boldsymbol{w})]$ (Προσεταιριστικότητα)
				\item $\forall\boldsymbol{v}\exists!\boldsymbol{0}(\boldsymbol{v} + \boldsymbol{0} = \boldsymbol{0} + \boldsymbol{v} = \boldsymbol{v})$ (Ουδέτερο στοιχείο πρόσθεσης)
				\item $\forall\boldsymbol{v}\exists!(-\boldsymbol{v})[\boldsymbol{v} + (-\boldsymbol{v}) = (-\boldsymbol{v}) + \boldsymbol{v} = \boldsymbol{0}]$ (Αντίθετος)
			\end{itemize}
			\item Βαθμωτός Πολλαπλασιασμός:
			\begin{itemize}
				\item $\forall\boldsymbol{v}\in V\forall a \in\mathbb{F}\exists!\boldsymbol{u}\in V(a\boldsymbol{v} = \boldsymbol{u})$ (Κλειστότητα)
				\item $\forall\boldsymbol{v}\in V\forall a,b \in\mathbb{F}[(ab)\boldsymbol{v} = a(b\boldsymbol{v})]$ (Προσεταιριστικότητα)
				\item ${\forall\boldsymbol{v}\in V(e_1\boldsymbol{v} = \boldsymbol{v})}$ (Ο βαθμωτός πολ/σμός με το ουδέτερο του πολ/σμού δεν μεταβάλλει το άνυσμα)
			\end{itemize}
		\end{enumerate}
		
		Eπιπλέον, μεταξύ ανυσματικής πρόσθεσης και βαθμωτού πολλαπλασιασμού πρέπει να ισχύουν τα αξιώματα:
		\begin{itemize}
			\item $\forall a, b\in\mathbb{F}\forall\boldsymbol{v}\in V[(a+b)\boldsymbol{v} = a\boldsymbol{v} + b\boldsymbol{v}]$ (Επιμερισμός βαθμωτών)
			\item $\forall a \in\mathbb{F}\forall\boldsymbol{v},\boldsymbol{u}\in V[a(\boldsymbol{v} + \boldsymbol{u}) = a\boldsymbol{v} + a\boldsymbol{u}]$ (Επιμερισμός ανυσμάτων)
		\end{itemize}
	\end{definition}

	Οι πράξεις αυτές θυμίζουν πολύ τις πράξεις διανυσμάτων τα οποία ορίζονται στον χώρο $\mathbb{R}^3$. Και πράγματι, αυτή υπήρξε η βάση για την γενίκευση των διανυσμάτων. Αλλά ένα άνυσμα δεν χρειάζεται να είναι $n$-άδα αριθμών για να θεωρηθεί άνυσμα. Κάθε δομή στοιχείων του $\mathbb{F}$ στο οποίο είναι ορισμένος ο διανυσματικός χώρος αποτελεί διάνυσμα, αρκεί να έχουμε ορίσει τις δύο άνω πράξεις. Όπως και στην περίπτωση των σωμάτων, δεν είναι απαραίτητο οι πράξεις της πρόσθεσης και του βαθμωτού πολλαπλασιασμού να θυμίζουν τις πράξεις των τυπικών διανυσμάτων. Τα σύμβολα και οι ονομασίες είναι καθαρά τυπικές. 
	
	Ορισμένα παραδείγματα διανυσματικών χώρων είναι:
	\begin{itemize}
		\item $\mathbb{R}^n$: Τα ανύσματα του χώρου αυτού είναι $n$-άδες πραγματικών αριθμών που μπορούν να αναπαρασταθούν με πίνακες-στήλες.
		\item $\mathbb{C}^n$: Τα ανύσματα του χώρου αυτού θυμίζουν τα προηγούμενα, αλλά είναι  $n$-άδες μιγαδικών, αυτήν την φορά, αριθμών. 
		\item ${P^\mathbb{F}_n \equiv \left\{\sum_{i=0}^na_ix^i : i\leq n, x\in\mathbb{F}\right\}}$: Τα ανύσματα του χώρου αυτού είναι πολυώνυμα τάξης \textbf{το πολύ} n. Αυτός ο χώρος είναι ένα άμεσο παράδειγμα του πόσο αφηρημένη είναι η έννοια του ανύσματος. Δεν χρειάζεται να είναι $n$-άδα αριθμών. H διανυσματική πρόσθεση είναι η ${R(x) = P(x) + Q(x)}$ και ο βαθμωτός πολ/σμός ο $Q(x) = aP(x)$.
		\item ${C[a, b] \equiv \left\{f:[a,b]\rightarrow\mathbb{R}:f \;\text{συνεχής}\right\}}$: Τα ανύσματα του χώρου αυτού είναι όλες οι συνεχείς πραγματικές συναρτήσεις με πεδίο ορισμού το $[a, b] \subset \mathbb{R}$. Όπως και με τα πολυώνυμα, αποτελούν ένα άλλο παράδειγμα αφηρημένου διανύσματος, τα οποία είναι μάλιστα \textit{απειροδιάστατα} (η διάσταση θα εξηγηθεί ύστερα). Οι πράξεις μεταξύ διανυσμάτων ορίζονται όπως και με τα πολυώνυμα.
		\item $\mathbb{F}$: Ένα αλγεβρικό σώμα είναι το ίδιο διανυσματικός χώρος, όπως φαίνεται πολύ απλά: η πρόσθεση στα σώματα είναι πανομοιότυπη με την διανυσματική πρόσθεση και η πράξη του πολλαπλασιασμού επίσης ικανοποιεί τα αξιώματα του βαθμωτού πολλαπλασιασμού (δεν μας πειράζει το γεγονός ότι ο πολλαπλασιασμός στα σώματα έχει και αντιστρόφους. Τα αξιώματα των διανυσματικών πράξεων δεν μας απαγορεύουν την ιδιότητα αυτήν).
	\end{itemize}

	\begin{definition}
	Κάθε σύνολο ${U \subseteq V}$ ονομάζεται \textbf{υπόχωρος} (subspace) του $V$, αν τα στοιχεία του τηρούν το αξίωμα της κλειστότητας.
	\end{definition}

	Τώρα λοιπόν έχουμε μια αρχική διαίσθηση του τι πα να πει διανυσματικός χώρος. Γενικά, βλέπουμε ότι η αφηρημένη έννοια έχει «εμπνευσθεί» από τα διανύσματα τα οποία γνωρίζουμε από τον λογισμό πολλών μεταβλητών. Αλλά οι ομοιότητες δεν παύουν εκεί!
	
	\begin{remark}
		Για βραχύτητα, τα ανύσματα είναι πάντοτε γραμμένα με έντονα γράμματα και τα βαθμωτά με αχνά. Αν δεν αναφέρομαι πού ανήκει ένα άνυσμα, τότε εννοώ ότι ανήκει στον «κύριο» χώρο στον οποίο ορίζεται. Προφανώς, τα βαθμωτά ανήκουν πάντοτε στο σώμα επί του οποίου ορίζεται ο διανυσματικός χώρος. Επίσης, τα ανύσματα του χώρου $\mathbb{R}^n$ θα αποκαλούνται διανύσματα για διαφοροποίηση, αλλά γενικά δεν υπάρχει διαφορά μεταξύ των όρων «άνυσμα» και «διάνυσμα»
	\end{remark}
	
	\newpage
	\subsection{Γραμμικοί συνδυασμοί, βάσεις, διαστάσεις, συντεταγμένες}
	
	\begin{definition}
	Ονομάζουμε ένα άνυσμα ${\boldsymbol{u} \in V}$ \textbf{γραμμικό συνδυασμό} (linear combination) ανυσμάτων τoυ χώρου $V$ αν εκφράζεται με σχέση της μορφής:
		\begin{align*}
			\boldsymbol{u} = \sum^n_{i=0}a_i\boldsymbol{v}_i
		\end{align*}
	\end{definition}
	Παράδειγμα:
	\begin{itemize}
		\item $\mathbb{R}^2$: Το διάνυσμα $\boldsymbol{u} = (8, 11)$ είναι γραμμικός συνδυασμός των διανυσμάτων $\boldsymbol{v}_1 = (3, 5)$ και $\boldsymbol{v}_2 = (2, 1)$ ($\boldsymbol{u} = 2\boldsymbol{v}_1 + \boldsymbol{v}_2$)
		\item $P^\mathbb{C}_3$: To πολυώνυμο $\boldsymbol{P}(z) = 5iz^2 + 3z^3$ είναι γραμμικός συνδυασμός των πολυωνύμων $\boldsymbol{Q}_1(z) = -2i + (5/3)z + z^2$ και $\boldsymbol{Q}_2 = 3 + (i-1)z^2$ ($\boldsymbol{P}(z) = 3i\boldsymbol{Q}_1(z) - 2\boldsymbol{Q}_2(z)$)
	\end{itemize}

	\begin{definition}
		Αν για ένα σύνολο ανυσμάτων ${S \subset V}$ ισχύει:
		\begin{align*}
			\forall \boldsymbol{v} \in V(\boldsymbol{v} \; \text{γραμμικός συνδυασμός των} \; \boldsymbol{u}\in S)
		\end{align*}
		Τότε ο χώρος $V$ \textbf{απλώνεται} (is spanned) από τα ανύσματα του $S$. Αντιστρόφως, o υπόχωρος
		\begin{align*}
			U \equiv \left\{\sum^n_{i=0}a_i\boldsymbol{v}_i\right\} \subseteq V
		\end{align*}
		ονομάζεται \textbf{άπλωμα} (span) των $\boldsymbol{v} \in S$.
	\end{definition}
	
	\begin{definition}
		Αν για ένα σύνολο ανυσμάτων $S \subset V$ ισχύει ότι:
		\begin{align*}
			\sum_{i=0}^na_i\boldsymbol{v}_i = \boldsymbol{0} \Rightarrow (\forall i \leq n)(a_i = 0)
		\end{align*}
		Τότε τα ανύσματα του $S$ ονομάζονται \textbf{γραμμικώς ανεξάρτητα} (linearly independent).
	\end{definition}

	Διαισθητικά, η γραμμική ανεξαρτησία σημαίνει ότι τα ανύσματα του $S$ δεν είναι πάνω στην ίδια «ευθεία».
	Παράδειγμα:
	\begin{itemize}
		\item $\mathbb{R}^3$: Τα διανύσματα $(4, 6, 0)$ και $(0, 3, 12)$ είναι προφανώς γραμμικώς ανεξάρτητα. 
		\item $P^\mathbb{C}_2$: Τα μονώνυμα $z^2$ και $z$ είναι γραμμικώς ανεξάρτητα, αφού δεν υπάρχουν σταθερές ${a,b \neq 0}$ τέτοιες ώστε $(\forall z \in \mathbb{C})(az + bz^2 = 0)$
	\end{itemize}
	
	\begin{definition}
		Έστω σύνολο ${S \subset V}$. Αν ισχύει ότι το $S$ είναι άπλωμα του $V$ και επίσης αποτελείται μόνο από γραμμικώς ανεξάρτητα ανύσματα, τότε το $S$ ονομάζεται \textbf{βάση} (base) του $V$.
	\end{definition}
	Τα διανύσματα μιας βάσης θα συμβολίζονται ως $\boldsymbol{e}_i$, όπου $i \in \mathbb{N}$ δείκτης.\\
	Παραδείγματα βάσεων:
	\begin{itemize}
		\item $\mathbb{R}^3$: Τα διανύσματα $\boldsymbol{e}_1 = (5, 0, 0)$, $\boldsymbol{e}_2 = (0, 2, 0)$, $\boldsymbol{e}_3 = (0, 0, 7)$ είναι βάση του χώρου $\mathbb{R}^3$
		\item $C[a,b]$: Προφανώς, όλες οι συναρτήσεις $\boldsymbol{e}_n(x) = \sin(nx), n\in \mathbb{N}$ είναι γραμμικώς ανεξάρτητες μεταξύ τους και από ανάπτυγμα Fourier, ξέρουμε ότι μπορούμε να εκφράσουμε όλες τις συναρτήσεις του χώρου $C[a,b]$ ως άπειρα αθροίσματα των $\boldsymbol{e}_n$. Ως εκ τούτου, το σύνολο $S = \left\{\boldsymbol{e}_n : n \in \mathbb{N}\right\}$ είναι βάση του χώρου $C[a,b]$.
	\end{itemize}
	Η βάση ενός χώρου προφανώς δεν είναι μοναδική.
	
	\newpage
	
	\begin{definition}
		Έστω ${S \subset V}$ βάση του χώρου $V$. Η πληθικότητα (ή πληθάριθμος) $|S|$ του $S$ ονομάζεται \textbf{διάσταση} (dimension) του χώρου $V$ και συμβολίζεται $\dim(V)$.
	\end{definition}
	
	\begin{theorem}
		Όλες οι βάσεις του χώρου $V$ έχουν ίση πληθικότητα. (\textit{αποδεικνύεται επαγωγικά})
 	\end{theorem}
 	Δηλαδή ένας χώρος δεν γίνεται να έχει δύο ή περισσότερες διαφορετικές διαστάσεις. Έχει μοναδική διάσταση.

 	Αν ισχύει:
 	\begin{itemize}
 		\item ${\exists n \in \mathbb{N}(\dim(V) = n)}$, τότε ο χώρος $V$ είναι πεπερασμένης διάστασης. Παράδειγμα ο χώρος $\mathbb{R}^4,$ ο οποίος είναι τετραδιάστατος ή ο χώρος $P^\mathbb{C}_n$, ο οποίος έχει ${n+1}$ διαστάσεις, όπου $n$ η μέγιστη δυνατή τάξη που μπορούν να έχουν τα πολυώνυμα του χώρου.
 		\item $\dim(V) = \aleph_0 = \beth_0$, ο χώρος είναι απειροδιάστατος, αλλά με διακριτές διαστάσεις. Παράδειγμα τέτοιου χώρου $C[a,b]$, δηλαδή ο χώρος των συνεχών πραγματικών διαστάσεων σε πεπερασμένο χωρίο του $\mathbb{R}$.
 		\item ${\dim(V) = \beth_1}$, o χώρος είναι απειροδιάστατος αλλά με συνεχείς διαστάσεις. Σε τέτοιους χώρους, δηλαδή, μπορεί κάποιος να μιλήσει, για παράδειγμα, για την $\pi$-οστή διάσταση. Παράδειγμα τέτοιου χώρου είναι ο $C[-\infty,+\infty]$, ο χώρος των συνεχών πραγματικών διαστάσεων ορισμένων σε όλον τον $\mathbb{R}$. Οι γραμμικοί συνδυασμοί σε τέτοιους χώρους αλλάζουν από διακριτά αθροίσματα σε ολοκληρώματα (δηλαδή συνεχή αθροίσματα).
 	\end{itemize}
 
 	Όταν έχουμε μία βάση $S$ του χώρου $V$, τότε για έναν γραμμικό συνδυασμό των στοιχείων του $S$ είναι κουραστικό να γράφουμε κάθε φορά την σχέση του ορισμού $2.1.1$. Ως πρώτη σκέψη, φαίνεται ότι μπορούμε να φτιάξουμε ένα σύνολο με τους συντελεστές του γραμμικού αυτού του συνδυασμού, ως κάποιου είδους «συντομογραφία». Το πρόβλημα είναι πως τα σύνολα εξ ορισμού δεν έχουν κάποιου είδους οργάνωση. Αυτό σημαίνει ότι αν είχαμε δύο γραμμικούς συνδυασμούς όπως κάτωθι:
 	\begin{align*}
 		\boldsymbol{u}_1 &= 3\boldsymbol{e}_1 + 6\boldsymbol{e}_2 + 7\boldsymbol{e}_3\\
 		\newline\\
 		\boldsymbol{u}_2 &= 7\boldsymbol{e}_1 + 3\boldsymbol{e}_2 + 6\boldsymbol{e}_3
 	\end{align*}
 	τότε το σύνολο $\{3, 6, 7\}$ εκφράζει και τα δύο ανύσματα ($\{3,6,7\} = \{7,3,6\}$). Ως εκ τούτου, θέλουμε να θέσουμε μία σειρά στα στοιχεία της βάσης $S$, έτσι ώστε να έχει νόημα μια τέτοια συντομογραφία.
 	
	\begin{definition}
		Έστω βάση $S$ του χώρου $V$. Αν ορίσουμε μία σχέση ${< \;\subset S \times S}$, μεταβατική και καλώς ορισμένη ώστε να ισχύει $\boldsymbol{e}_1 < \boldsymbol{e}_2 < \ldots < \boldsymbol{e}_n < \ldots, \; (\forall i < \dim(V))(\boldsymbol{e}_i \in S)$, τότε για έναν γραμμικό συνδυασμό
		\begin{align*}
			\boldsymbol{u} = \sum^{dim(V)}_{i=0}a_i\boldsymbol{e}_i
		\end{align*}
		τα στοιχεία της πλειάδας $(a_1, a_2, \ldots, a_n, \ldots)_S$ ονομάζονται \textbf{συντεταγμένες} (coordinates) του $\boldsymbol{u}$ στην βάση $S$. Η $S$ θα αποκαλείται \textbf{διατεταγμένη} (ordered).
	\end{definition}
	\begin{remark}
		Από εδώ και στο εξής, θα υπονοείται ότι κάθε βάση είναι διατεταγμένη. Η σειρά με την οποία αναγράφονται τα στοιχεία της βάσης είναι και η εννοούμενη διάταξη.
	\end{remark}
	\newpage
		Έστω διατεταγμένη βάση $S = \{e_i\}$ του χώρου $V$.
	\begin{theorem}
		Το σύνολο ${S' = \{(\boldsymbol{e}_i + \boldsymbol{u}_i)\}}$ είναι επίσης βάση του χώρου $V$, αν ${\det{(U)}\neq 0}$, όπου $U$ ο πίνακας με στοιχεία ${U_{ji} = u_{ij} + \delta_{ij}}$, ${u_{ij}}$ οι συντεταγμένες των $\boldsymbol{u}_i$ και $\delta_{ij}$ το δέλτα του Κρόνεκερ.
	\end{theorem}
	Για τα διανύσματα του χώρου $\mathbb{R}^3$, όπως και να στρίψουμε τα διανύσματα μιας βάσης, θα αποτελούν και πάλι βάση, αρκεί να μην πέσουν τουλάχιστον δύο σε μια ευθεία που περνά από την αρχή των αξόνων. Αυτή η ιδέα γενικεύεται, λοιπόν, με το ανωτέρω θεώρημα και σε ανύσματα άλλων χώρων. Παράδειγμα εφαρμογής του θεωρήματος:
	\begin{align*}
		&\text{Στον χώρο}\; \mathbb{R}^2,\; \text{έστω η βάση}\; S=\{(1,0),\; (0,1)\}.\\
		&\text{Έστωσαν διανύσματα}\; \boldsymbol{u}_1 = (5, 3), \; \boldsymbol{u}_2 = (-3, 8)\\
		&U = \begin{pmatrix}
			5 + 1 & 3 \\
			-3 & 8 + 1
		\end{pmatrix} =
		\begin{pmatrix}
			6 & 3 \\
			-3 & 9
		\end{pmatrix}\\
		&\det(U) = 6\cdot9 + 9 = 35 \neq 0\\
		&\text{Άρα το σύνολο}\; S' = \{(1,0) + \boldsymbol{u}_1,\; (0,1) + \boldsymbol{u}_2\} =\\
		&\foreach \i in {1, 2, 3, 4, 5, 6, 7, 8, 9, 10}{\quad} = \{(6,3),\;(-3,9)\}\\
		&\text{είναι επίσης βάση.}\\
		\newline\\
		&\text{Από την άλλη, έστωσαν τα διανύσματα}\; \boldsymbol{u}'_1 = (0,1)\; \boldsymbol{u}'_2 = (2,1) \\
		&U' = \begin{pmatrix}
			0 + 1 & 1 \\
			2 & 1 + 1
		\end{pmatrix} =
		\begin{pmatrix}
			1 & 1 \\
			2 & 2
		\end{pmatrix}\\
		&\text{Προφανώς}\; \det(U') = 0\\
		&\text{Άρα}\; S'' = \{(1,0) + \boldsymbol{u}'_1,\; (0,1) + \boldsymbol{u}'_2\} =\\
		&\foreach \i in {1, 2, 3, 4}{\quad} = \{(1,1),\;(2,2)\}\; \text{δεν είναι βάση.}\\
		\newline\\
		&\text{ΠΡΟΣΟΧΗ: Αν κάποιος αντιστρέψει τα στοιχεία}\\
		&\text{της βάσης (} S = \{(0,1)\;(1,0)\}\text{)}\\
		&\text{η}\;S^{(3)} = \{(0,1) + \boldsymbol{u}'_1,\;(1,0) + \boldsymbol{u}'_2\}\; \text{είναι βάση.}\\
		&\text{Αλλά τότε αλλάζει και ο πίνακας}\;U\;\text{και ισχύει}\\
		&\det(U) \neq 0\text{ αφού οι συντεταγμένες των } \boldsymbol{u}'_i \; \text{θα αλλάξουν.}\\
		&\text{Δηλαδή δεν παύει να ισχύει το θεώρημα.}
	\end{align*}
	
	\begin{corollary}
		To σύνολο ${S' = \{b_i\boldsymbol{e}_i\}}$ είναι βάση του χώρου $V$, αν ${\forall i(b_i \neq 0)}$.
	\end{corollary}
	Διαισθητικά μιλώντας, η «αλλαγή κλίμακας» δίνει πάλι βάση. Παράδειγμα:
	\begin{align*}
		\text{Στον χώρο}\;&\mathbb{R}^3, \text{έστω η βάση} \; S = \{(1,0,0),\; (0,1,0),\; (0,0,1)\}. \\
		\text{To σύνολο}\; &S'=\{(5, 0, 0),\;(0,3,0),\;(0,0,-10)\} \; \text{είναι επίσης βάση.}
	\end{align*}
	\newpage
	\subsection{Νόρμα, βασική τοπολογία, χώροι Banach}
	
	Μια άλλη χρήσιμη ιδιότητα του Ευκλείδειου χώρου που θα θέλαμε να γενικεύσουμε είναι αυτή της απόστασης και του μέτρου.
	
	\begin{definition}
		Έστω ανυσματικός χώρος $V$ επί σώματος $\mathbb{F}$. Ορίζουμε συνάρτηση $\|\cdot\|: V \rightarrow \mathbb{R}$, η οποία τηρεί τα κάτωθι αξιώματα:
		\begin{itemize}
			\item $\forall \boldsymbol{u}[(\|\boldsymbol{u}\| \geq 0)\wedge(\|\boldsymbol{u}\| = 0 \Leftrightarrow \boldsymbol{u} = \boldsymbol{0})]$, η συνάρτηση δίνει μη αρνητική ποσότητα και μηδενική ποσότητα ΜΟΝΟ για το μηδενικό άνυσμα.
			\item $\forall \boldsymbol{u}(\|\boldsymbol{u}\| < \infty)$, η συνάρτηση δεν απειρίζεται για κανένα άνυσμα. 
			\item $\forall a,\boldsymbol{u}(\|a\boldsymbol{u}\| = |a|\|\boldsymbol{u}\|)$, το βαθμωτό στοιχείο «βγαίνει απέξω» (το μέτρο αυτού).
			\item $\forall \boldsymbol{u},\boldsymbol{v}(\|\boldsymbol{u} + \boldsymbol{v}\| \leq \|\boldsymbol{u}\| + \|\boldsymbol{v}\|)$, ισχύει η τριγωνική ανισότητα.
		\end{itemize}
		Η συνάρτηση αυτή ονομάζεται \textbf{νόρμα} (norm) του χώρου $V$.
	\end{definition}
	Αν για μια συνάρτηση δεν ισχύει ότι $\|\boldsymbol{u}\| = 0 \Leftrightarrow \boldsymbol{u} = \boldsymbol{0}$, αλλά ικανοποιεί όλα τα υπόλοιπα αξιώματα της νόρμας, ονομάζεται \textbf{ημινόρμα} (seminorm)
	\begin{remark}
		Όπως και με τους όρους «άνυσμα»-«διάνυσμα», οι όροι «μέτρο» και «νόρμα» είναι πανομοιότυποι. Αλλά, για διαφοροποίηση, ο όρος «μέτρο» θα χρησιμοποιείται κατ' αποκλειστικότητα για τα διανύσματα του χώρου $\mathbb{R}^n$, αλλά και για τα στοιχεία σωμάτων, ενώ ο όρος «νόρμα» για ένα γενικό άνυσμα.
	\end{remark}

	Παραδείγματα νόρμας:
	\begin{itemize}
		\item $\mathbb{R}^n$: Για ένα διάνυσμα ${\boldsymbol{u} = (a_1, a_2, \ldots, a_n)}$, η συνάρτηση $\|\boldsymbol{u}\| = (a_1^2 + a_2^2 + \ldots + a_n^2)^{1/2}$ ικανοποιεί τα αξιώματα της νόρμας και άρα αποτελεί μέτρο του χώρου.
		\item $C[a,b]$: To ολοκλήρωμα ${\|\boldsymbol{f}(x)\| = \left|\int^b_af^2(x)dx\right|}$ ικανοποιεί τα αξιώματα της νόρμας και άρα αποτελεί νόρμα του χώρου.
		\item $C[-\infty, +\infty]$: To ολοκλήρωμα ${\|\boldsymbol{f}(x)\| =  \left|\int^{+\infty}_{-\infty}f^2(x)w(x)dx\right|}$, όπου $w(x)$ μια \textit{συνάρτηση βάρους} καταλλήλως ορισμένη έτσι ώστε να μην απειρίζεται το ολοκλήρωμα για καμμία $f(x)$, ικανοποιεί τα αξιώματα της νόρμας και άρα αποτελεί νόρμα του χώρου.
	\end{itemize}

	\begin{corollary}
		Η νόρμα ενός χώρου δεν είναι μοναδική.
	\end{corollary}

	Τώρα που έχουμε λάβει μια ιδέα για το πόσο «μακρύ» είναι ένα άνυσμα του χώρου μας, είναι χρήσιμο να ξέρουμε πόσο απέχουν δύο σημεία του χώρου.

	\begin{definition}
		Έστω νόρμες $\|\cdot\|_1$ και $\|\cdot\|_2$. Αν ισχύει ότι $\|\cdot\|_1 \sim \|\cdot\|_2$, δηλαδή:
		\begin{align*}
			\exists m,M \in \mathbb{R}^+\forall \boldsymbol{u}(m\|\boldsymbol{u}\|_2 \leq \|\boldsymbol{u}\|_1 \leq M\|\boldsymbol{u}\|_2)
		\end{align*}
		τότε λέμε ότι οι δύο νόρμες είναι \textbf{ανάλογες} (equivalent).
	\end{definition}

	\begin{definition}
		Ονομάζουμε \textbf{μετρική} (metric) ενός χώρου $V$ την συνάρτηση ${\dist_V: V\times V \rightarrow \mathbb{R}}$ με έκφραση:
		\begin{align*}
			\dist_V(\boldsymbol{v},\boldsymbol{u}) = \|\boldsymbol{v} - \boldsymbol{u}\|
		\end{align*}
		όπου $\|\cdot\|$ μία νόρμα του $V$.
	\end{definition}
	Όπως και με την νόρμα, η μετρική δεν είναι μοναδική. Κανονικά, η μετρική δεν είναι απαραίτητο να ορισθεί μέσω της νόρμας. Μάλιστα, υπάρχουν χώροι με μετρική (μετρικοί χώροι), στους οποίους μπορεί να μην μπορεί να ορισθεί νόρμα. Προφανώς, ένας χώρος με νόρμα, από την άλλη, έχει πάντοτε και μετρική. Την μετρική που ορίζεται μέσω νόρμας την ονομάζουμε συγκεκριμένα \textbf{μετρική εκ μέτρου επαγόμενη} (norm induced metric).
	
	\begin{theorem}
		Η μετρική ενός ανυσματικού χώρου $V$ είναι συμμετρική. Δηλαδή:
		\begin{align*}
			\forall\boldsymbol{u},\boldsymbol{v}(\dist_V(\boldsymbol{u}, \boldsymbol{v}) = \dist_V(\boldsymbol{v}, \boldsymbol{u}))
		\end{align*}
	\end{theorem}

	Μια έννοια που χρησιμοποιούμε συχνά αλλά ανεπαίσθητα στην Φυσική είναι αυτή της ισομετρίας.
	\begin{definition}
		\textbf{Ισομετρία} (isometry) ονομάζεται κάθε «1 προς 1» και επί συνάρτηση $f$ από μετρικό χώρο $V$ μετρικής $\dist_V$ σε μετρικό χώρο $W$ μετρικής $\dist_W$ για την οποία ισχύει:
		\begin{align*}
			(\forall \boldsymbol{v}_1,\boldsymbol{v}_2\in V)[\dist_V(\boldsymbol{v}_1, \boldsymbol{v}_2) = \dist_W(f(\boldsymbol{v}_1), f(\boldsymbol{v}_2))]
		\end{align*} 
	\end{definition}
	Με άλλα λόγια, η ισομετρία είναι μια συνάρτηση που διατηρεί την απόσταση μεταξύ των σημείων της περιοχής στην οποία δρα. Πώς σχετίζεται όμως η έννοια αυτή με την Φυσική; Η απάντηση είναι τα σχετικά διανύσματα. Θα γίνει αναφορά συγκεκριμένα στην Νευτώνεια Μηχανική.
	
	Όταν θέλουμε να μελετήσουμε τον χώρο ως προς ένα νέο σημείο αυτού, πρέπει να κάνουμε κάποιου είδους «αλλαγή» στον χώρο. Η πρώτη σκέψη θα ήταν ο μετασχηματισμός της βάσης του εν λόγω χώρου. Όμως, αυτό δεν είναι δυνατόν· ο Ευκλείδειος χώρος, τον οποίο χρησιμοποιεί η Νευτώνεια Μηχανική, αποτελείται από διανύσματα που λαμβάνουν την μορφή τριάδων πραγματικών αριθμών. Δεν υπάρχει τρόπος να κωδικοποιήσουμε στις τριάδες αυτές καθ' αυτές την ιδέα ότι η αρχή των αξόνων είναι ένα άνυσμα πέρα της τριάδας $(0,0,0)$. Αντ' αυτού, αλλάζουμε την θέση των πάντων στον χώρο χρησιμοποιώντας μια ισομετρία $f: \mathbb{R}^3 \rightarrow \mathbb{R}^3$. Για παράδειγμα, αν θέλουμε να μελετήσουμε τον χώρο ως προς το σημείο $(4,2,9)$, πρέπει να χρησιμοποιήσουμε την ισομετρία:
	\begin{align*}
		f(\boldsymbol{x}) = \boldsymbol{x} - (4,2,9)
	\end{align*}

	\begin{remark}
		Στο επόμενο χωρίο θα αναφέρω ορισμένους βασικούς και στοιχειώδεις όρους τοπολογίας, καθώς θεωρώ είναι απαραίτητοι για την πλήρη γνώση των ανυσματικών χώρων. Δεν είναι απαραίτητο η τοπολογία να περιορίζεται σε ανυσματικούς χώρους. Κάθε σύνολο με μετρική είναι «δόκιμο». Για συνέπεια, όμως, θα περιοριστούμε στους ανυσματικούς χώρους.
	\end{remark}

	Στους ανυσματικούς χώρους με μετρική δυνάμεθα να ορίσουμε έννοιες παρόμοιες με τα διαστήματα του $\mathbb{R}$:
	\begin{definition}
		Σε χώρο $V$ με μετρική $\dist_V$, ορίζουμε τα σύνολα:
		\begin{itemize}
			\item $B(\boldsymbol{u}_o, r) = \left\{\boldsymbol{u} : \dist_V(\boldsymbol{u}_o,\boldsymbol{u}) < r\right\}$: \textbf{Ανοιχτή μπάλα} (Open ball) κέντρου $\boldsymbol{u}_o$ και ακτίνας $r$
			\item $\bar{B}(\boldsymbol{u}_o, r) = \left\{\boldsymbol{u} : \dist_V(\boldsymbol{u}_o,\boldsymbol{u}) \leq r\right\}$: \textbf{Κλειστή μπάλα} (Closed ball) κέντρου $\boldsymbol{u}_o$ και ακτίνας $r$
			\item $S(\boldsymbol{u}_o, r) = \left\{\boldsymbol{u} : \dist_V(\boldsymbol{u}_o,\boldsymbol{u}) = r\right\}$: \textbf{Σφαίρα} (Sphere) κέντρου $\boldsymbol{u}_o$ και ακτίνας $r$
		\end{itemize}
		όπου $r > 0$
	\end{definition}
	Με άλλα λόγια, μία σφαίρα ορίζεται όπως ακριβώς και στην Ευκλείδια γεωμετρία: όλα τα σημεία ενός χώρου που ισαπέχουν από ένα κέντρο. Η ανοιχτή μπάλα είναι τα σημεία μέσα στην σφαίρα και η κλειστή μπάλα είναι η ένωση της σφαίρας και της ανοιχτής μπάλας.
	
	Μπορούμε τις έννοιες αυτές να τις επεκτείνουμε σε ένα σύνολο $D\subseteq V$ «τυχαίου σχήματος».
	
	\begin{definition}
		Ονομάζουμε \textbf{ανοιχτό σύνολο} (open set) ένα σύνολο ${D \subseteq V}$, όπου $V$ ανυσματικός χώρος με μετρική, αν ισχύει:
		\begin{align*}
			\forall\boldsymbol{u}\in D:\exists\epsilon > 0[B(\boldsymbol{u},\epsilon)\subseteq D]
		\end{align*}
		Δηλαδή αν υπάρχει μία \textbf{ε-γειτονία} με κέντρο το $\boldsymbol{u}$ η οποία να ανήκει στο $D$.
	\end{definition}

	\begin{definition}
		Ονομάζουμε \textbf{κλειστό σύνολο} (closed set) ένα σύνολο ${D \subseteq V}$ αν το συμπληρωματικό του είναι ανοικτό.
	\end{definition}

	Παράδειγμα: Έστω ο χώρος $\mathbb{R}^2$ με μετρική $\|(a_1, a_2)\| = (a_1^2 + a_2^2)^{1/2}$.
	\begin{itemize}
		\item O ίδιος ο χώρος είναι ανοιχτό σύνολο. Κάθε μπάλα με κέντρο οποιοδήποτε σημείο του $\mathbb{R}^2$ είναι προφανώς υποσύνολο του $\mathbb{R}^2$.
		\item Το κενό σύνολο $\emptyset$ είναι κλειστό σύνολο, αφού το συμπληρωματικό του, ο χώρος $\mathbb{R}^2$ είναι ανοιχτό, όπως αναγράφεται άνω.
	\end{itemize}
	Ενδιαφέρον είναι πως για τα άνω παραδείγματα, ισχύει ότι το $\emptyset$ είναι επίσης ανοιχτό, αφού δεν ανήκει κανένα στοιχείο σε αυτό και άρα δεν υπάρχει στοιχείο τέτοιο ώστε να μην ικανοποιεί την συνθήκη του ανοιχτού συνόλου. Ως εκ τούτου, το $\mathbb{R}^2$ είναι επίσης κλειστό.
	
	Προφανώς, αυτός ο ορισμός του κλειστού συνόλου δεν είναι πάντοτε εύχρηστος. Έτσι, χρειάζεται να σχηματίσουμε έναν νέο «ορισμό», ο οποίος είναι καλύτερος στην χρήση.
	\begin{definition}
		Ονομάζουμε το σημείο $\boldsymbol{u}$ \textbf{σημείο συσσώρευσης} (accumulation point) ενός συνόλου ${D \subseteq V}$ αν κάθε γειτονία του $\boldsymbol{u}$ περιέχει ένα σημείο του $D$. Με μαθηματική γραφή:
		\begin{align*}
			(\boldsymbol{u}\;\text{σημείο συσσώρευσης του}\;D) \Leftrightarrow [\forall\epsilon>0:\exists\boldsymbol{v}\in D(\boldsymbol{v}\in B(\boldsymbol{u},\epsilon))]
		\end{align*}
		To σύνολο όλων των σημείων συσσώρευσης του $D$ το ονομάζουμε \textbf{κλειστότητα} (closure) του $D$ και συμβολίζεται $\bar{D}$
	\end{definition}
	Με την έννοια του σημείου συσσώρευσης, μπορούμε πλέον να δώσουμε έναν ισοδύναμο ορισμό του κλειστού συνόλου.
	\begin{theorem}
		Ένα σύνολο $D$ είναι κλειστό αν και μόνο αν $\bar{D} = D$
	\end{theorem}
	
	\begin{definition}
		Ένα σύνολο $D$ ονομάζεται \textbf{συμπαγές} (dense) σε έναν ανυσματικό χώρο $V$ αν και μόνο αν ${\bar{D} = V}$
	\end{definition}
	Με άλλα λόγια, ένα συμπαγές σύνολο ενός χώρου $V$ είναι αυτό του οποίου δεν του λείπουν παρά μόνο σημεία (και γενικότερα δομές μικρότερης διάστασης από τον ίδιο τον χώρο) από τον χώρο $V$.
	\begin{definition}
		Ένας ανυσματικός χώρος είναι \textbf{διαχωρίσιμος} (separable) αν υπάρχει τουλάχιστον ένα συμπαγές αλλά και αριθμήσιμο ($|D| \leq \aleph_0$) υποσύνολο.  
	\end{definition}
	Παράδειγμα: Οι πραγματικοί αριθμοί $\mathbb{R}$ είναι διαχωρίσιμοι και το υποσύνολό των ρητών αριθμών $\mathbb{Q}$ είναι συμπαγές στους πραγματικούς αριθμούς. Αυτό διαφαίνεται εύκολα, καθώς κάθε ακολουθία ρητών συγκλίνει σε έναν πραγματικό αριθμό και άρα κάθε πραγματικός αριθμός έχει τουλάχιστον έναν ρητό αριθμό οσοδήποτε κοντά του. Αυτό κάνει τους ρητούς αριθμούς ένα συμπαγές σύνολο, στους πραγματικούς αριθμούς, και, αφού είναι αριθμήσιμοι οι ρητοί, σύμφωνα με τον ορισμό της διαχωρισημότητας, τους πραγματικούς αριθμούς διαχωρίσιμους.
	
	\newpage
	Στην τοπολογία, υπάρχει μία έννοια που θυμίζει αυτήν του δυναμοσυνόλου.
	\begin{definition}
		Ονομάζουμε $\mathfrak{T}$ \textbf{τοπολογικό χώρο} (topological space) του ανυσματικού χώρου $V$ το σύνολο:
		\begin{align*}
			\mathfrak{T} = \left\{D\subseteq V : D \;\text{ανοιχτό σύνολο}\right\}
		\end{align*} 
	\end{definition}

	Τώρα, θέλουμε να επεκτείνουμε τις ιδέες του ορίου, της σύγκλισης ακολουθιών και της συνέχειας από την ανάλυση πραγματικών συναρτήσεων σε έναν γενικό ανυσματικό χώρο.
	\begin{definition}
		Ονομάζουμε το σημείο ${\boldsymbol{u} \in D}$ \textbf{εσωτερικό} (interior point) ενός συνόλου ${D \subseteq V}$ αν ισχύει ότι υπάρχει ανοικτό σύνολο ${M \subseteq D}$ το οποίο να περιέχει το $\boldsymbol{u}$.
		
		To σύνολο όλων των εσωτερικών σημείων του $D$ θα συμβολίζεται $\intr(D)$.
	\end{definition}

	\begin{definition}
		Έστω ακολουθία $\boldsymbol{u}_n$ ενός χώρου $V$ μετρικής $\dist_V$. Η ακολουθία αυτή \textbf{συγκλίνει} (converges) σε ένα σημείο $\boldsymbol{u}$ του $V$ αν και μόνο αν:
		\begin{align*}
			\forall\epsilon>0:\exists N = N(\epsilon):\forall n\geq N [\dist_V(\boldsymbol{u}_n, \boldsymbol{u}) < \epsilon]
		\end{align*}
		Θα συμβολίζουμε την σύγκλιση ως $\boldsymbol{u}_n \rightarrow \boldsymbol{u}$
	\end{definition}
	Με άλλα λόγια, μια ακολουθία ανυσμάτων συκλίνει σε ένα άνυσμα ενός ανυσματικού χώρου αν συγκλίνει η ακολουθία των αποστάσεών τους στο σημείο αυτό στο 0. Δηλαδή
	\begin{align*}
		(\boldsymbol{u}_n \rightarrow \boldsymbol{u}) \Leftrightarrow (a_n \rightarrow 0), \; \text{όπου}\; a_n = \dist_V(\boldsymbol{u}_n, \boldsymbol{u})
	\end{align*}

	Μια ακολουθία δεν είναι απαραίτητο να είναι ακολουθία ανυσμάτων, με την έννοια ότι πρέπει για κάθε $n$ να παίρνουμε συγκεκριμένο άνυσμα. Μπορεί αντ' αυτού, να έχουμε ακολουθία συναρτήσεων. Για αυτές τις ακολουθίες μπορούμε να ορίσουμε δύο ειδών συγκλίσεις:
	\begin{definition}
		Έστωσαν ακολουθία $f_n$ συναρτήσεων $f_n:X\rightarrow V$ και συνάρτηση $f:X\rightarrow V$ (δεν μας ενδιαφέρει το όρισμα των συναρτήσεων, παρά μόνον το ότι το αντιστοιχεί σε ένα άνυσμα του μετρικού χώρου $V$). Ορίζουμε τις εξής συγκλίσεις:
		\begin{align*}
			(f_n\; \text{συγκλίνει ομοιόμορφα στην}\;f) &\Leftrightarrow \forall\epsilon>0:\exists N = N(\epsilon):\forall x\in X:(\dist_V(f_n(x), f(x)) < \epsilon)\\
			(f_n\; \text{συγκλίνει κατά σημεία στην}\;f) &\Leftrightarrow \forall\epsilon>0:\forall x\in X:\exists N = N(\epsilon,x):(\dist_V(f_n(x), f(x)) < \epsilon)\\
		\end{align*}
	\end{definition}
	Αρχικά, αυτοί οι δύο ορισμοί φαίνονται πανομοιότυποι. Απλά αλλάξαμε την σειρά των προτάσεων του ορισμού. Αλλά αυτή η αλλαγή έχει τεράστια σημασία. Η ομοιόμορφη σύκλιση μας λέει ότι μια ακολουθία συναρτήσεων συγκλίνει σε μια συνάρτηση με τον ίδιο ρυθμό για όλες τις δυνατές τιμές του ορίσματός της, εξού και η ονομασία \textbf{ομοιόμορφη} (uniform). Από την άλλη, η \textbf{κατά σημεία σύγκλιση} (pointwise) έχει ασθενέστερα κριτήρια, και απλά απαιτεί ότι η κάθε ακολουθία ανυσμάτων $\boldsymbol{u}_n = f_n(c)$, όπου $c$ μια σταθερά που ανήκει στο $X$ θα συγκλίνει κάποτε στo άνυσμα $f(c)$. Η διαφορά αυτή μεταξύ ομοιόμορφης και κατά σημεία σύγκλισης είναι και ο λόγος που εμφανίζεται το φαινόμενο Gibbs στους μετασχηματισμούς Fourier.
	\newpage
	Μια άλλη ιδιότητα που μπορούμε να μεταφέρουμε από την ανάλυση είναι η έννοια των φραγμένων συνόλων.
	\begin{definition}
		Ένα υποσύνολο ενός χώρου $V$ με μετρική ονομάζεται \textbf{φραγμένο} (bounded) αν
		\begin{align*}
			\delta(D) = \underset{\boldsymbol{u},\boldsymbol{v}\in D}{\sup}[\dist_V(\boldsymbol{u}, \boldsymbol{v})] < \infty
		\end{align*}
		όπου $\delta(D)$ ονομάζεται \textbf{διάμετρος} (diameter) του $D$
	\end{definition}

	\begin{definition}
		Μία ακολουθία ονομάζεται \textbf{Cauchy} αν ισχύει:
		\begin{align*}
			\forall\epsilon>0:\exists N = N(\epsilon):\forall n,m\geq N [\dist_V(\boldsymbol{u}_n, \boldsymbol{u}_m) < \epsilon]
		\end{align*}
	\end{definition}
	Ο ορισμός αυτός θυμίζει πολύ τον ορισμό της σύγκλισης, αλλά λέει κάτι διαφορετικό. Μας λέει ότι οι ακολουθίες Cauchy είναι αυτές που καθώς πηγαίνουμε προς το άπειρο, τόσο λιγότερο αλλάζει η τιμή της ακολουθίας. Αλλά αυτό δεν υπονοεί σύγκλιση, καθώς η σύγκλιση μιας ακολουθίας σε έναν χώρο $V$ \textbf{προϋποθέτει} ότι το σημείο σύγκλισης βρίσκεται μέσα στον $V$. Παράδειγμα:\\
	Στον χώρο $\mathbb{Q}$ με μετρική ${\dist_\mathbb{Q}(p, q) = |p - q|}$, προφανώς υπάρχει μια ακολουθία ρητών $p_n$ που πλησιάζουν σε έναν άρρητο, έστω $\sqrt{2}$. Όμως, αυτός ο αριθμός δεν ανήκει στον $\mathbb{Q}$. Οπότε, δεν είναι ορθό να μιλήσουμε για σύγκλιση της ακολουθίας $p_n$ στον αριθμό αυτόν, αφού δεν γίνεται να οριστεί η μετρική των ρητών αριθμών με ένα εκ των δύο ανυσμάτων να είναι ο $\sqrt{2}$. Αλλά, η ακολουθία αυτή είναι μια ακολουθία \textbf{Cauchy}.
	
	Βέβαια, αυτό δεν σημαίνει ότι οι δύο ορισμοί είναι ασύμβατοι. Πράγματι υπάρχει μια σχέση μεταξύ τους:
	\begin{corollary}
		Έστω ακολουθία $\boldsymbol{u}_n$ σε χώρο $V$ με μετρική.
		\begin{align*}
			(\boldsymbol{u}_n \; \text{συγκλίνει στον}\; V) \Rightarrow (\boldsymbol{u}_n \; \text{είναι ακολουθία Cauchy})
		\end{align*}
	\end{corollary}
	Τι συμβαίνει όμως όταν ισχύει και το αντίστροφο; Τότε λέμε ότι ένας χώρος είναι πλήρης.
	\begin{definition}
		Ένας ανυσματικός χώρος $V$ αποκαλείται \textbf{πλήρης} (complete) αν και μόνο αν ισχύει ότι για κάθε ακολουθία $\boldsymbol{u}_n$ του χώρου αυτού
		\begin{align*}
			(\boldsymbol{u}_n\; \text{ακολουθία Cauchy}) \Rightarrow (\boldsymbol{u}_n\; \text{συγκλίνει στον}\; V)
 		\end{align*}
	\end{definition}
	Με άλλα λόγια, ένας χώρος είναι πλήρης άμα δεν έχει «τρύπες».
	Παράδειγμα: Η πραγματική ευθεία $\mathbb{R}$ είναι πλήρης καθώς κάθε συγκλίνουσα πραγματική ακολουθία που μπορούμε να σκεφτούμε θα συγκλίνει σε έναν πραγματικό αριθμό. Από την άλλη, η ευθεία των ρητών $\mathbb{Q}$ δεν είναι πλήρης, καθώς κάποιες ακολουθίες της μπορούν να συγκλίνουν μόνο σε πραγματικούς αριθμούς.

	\begin{definition}
		Ένας \textbf{χώρος Banach} είναι οποιοσδήποτε ανυσματικός χώρος διαθέτει νόρμα και είναι πλήρης.
	\end{definition}
	
	Μια πολύ χρήσιμη έννοια είναι αυτή του ορίου.
	\begin{definition}
		Ονομάζουμε \textbf{όριο} (limit) μιας συνάρτησης ${f:D\rightarrow W}$, όπου ${D\subseteq V}$, $V$ και $W$ ανυσματικοί χώροι με μετρική, σε \textbf{σημείο συσσώρευσης} ${\boldsymbol{v}_o\in V}$ του $D$, μια τιμή ${\boldsymbol{l}\in W}$, για την οποία ισχύει:
		\begin{align*}
			\forall\delta >0:\exists \epsilon >0 : \forall \boldsymbol{u}\in ([B(\boldsymbol{u}_o,\epsilon)\cap D]\setminus\{\boldsymbol{u}_o\})[\dist_W(f(\boldsymbol{u}),\boldsymbol{l}) < \delta]
		\end{align*}
		Θα συμβολίζουμε το όριο ως $\underset{\boldsymbol{u} \rightarrow \boldsymbol{u}_o}{lim}(f(\boldsymbol{u})) = l$
	\end{definition}
	
	\begin{definition}
		Μια συνάρτηση ${f:D\rightarrow W}$ από υποσύνολο $D$ μετρικού χώρου $V$ σε μετρικό χώρο $W$ αποκαλείται \textbf{συνεχής} (continuous) σε ένα σημείο συσσώρευσης $\boldsymbol{u}_o\in D$ του $D$ αν ισχύει:
		\begin{align*}
			\underset{\boldsymbol{u} \rightarrow \boldsymbol{u}_o}{lim}(f(\boldsymbol{u})) = f(\boldsymbol{u}_o)
		\end{align*}
	\end{definition}
	\newpage
	\subsection{Εσωτερικό Γινόμενο και χώροι Hilbert}
	Έχουμε σχεδόν τελειώσει με την πλήρη γεωμετρική περιγραφή ενός ανυσματικού χώρου. Όμως, δεν έχουμε μεταφράσει μία ακόμη έννοια της Ευκλείδιας γεωμετρίας· την γωνία. Αυτή η έννοια απαιτεί τον ορισμό του εσωτερικού γινομένου.
	
	\begin{definition}
		Ονομάζουμε \textbf{εσωτερικό γινόμενο} (inner product) ενός ανυσματικού χώρου $V$ μια συνάρτηση ${\langle\cdot,\cdot\rangle:V\times V \rightarrow \mathbb{F}}$ η οποία ικανοποιεί τα εξής αξιώματα:
		\begin{itemize}
			\item $\forall\boldsymbol{u}_1,\boldsymbol{u}_2,\boldsymbol{v}[\langle \boldsymbol{u}_1 + \boldsymbol{u}_2, \boldsymbol{v}\rangle = \langle\boldsymbol{u}_1,\boldsymbol{v}\rangle + \langle\boldsymbol{u}_2,\boldsymbol{v}\rangle]$
			\item $\forall\boldsymbol{u},\boldsymbol{v}:\forall a\in\mathbb{F}[\langle a\boldsymbol{u},\boldsymbol{v}\rangle = a\langle\boldsymbol{u},\boldsymbol{v}\rangle]$
			\item ${\forall\boldsymbol{u},\boldsymbol{v}[\langle\boldsymbol{u},\boldsymbol{v}\rangle = \langle\boldsymbol{v},\boldsymbol{u}\rangle^*]}$, το εσωτερικό γινόμενο δεν είναι συμμετρικό, αλλά η αντιστροφή των ορισμάτων δίνει τον συζυγή αριθμό του.
			\item $\forall\boldsymbol{u}\{(\langle\boldsymbol{u},\boldsymbol{u}\rangle \in \mathbb{R}^+)\wedge[(\langle\boldsymbol{u},\boldsymbol{u}\rangle = 0) \Leftrightarrow \boldsymbol{u} = \boldsymbol{0}]\}$, το εσωτερικό γινόμενο ενός ανύσματος με τον εαυτό του δίνει μη αρνητικό αριθμό κσι μηδέν αν και μόνο αν το διάνυσμα αυτό είναι το μηδενικό.
		\end{itemize}
	\end{definition}
	Τα πρώτα δύο αξιώματα υπονοούν ότι το εσωτερικό γινόμενο είναι μία διγραμμική απεικόνιση. Με άλλα λόγια, είναι γραμμικό ως προς κάθε όρισμά του. Όταν μια συνάρτηση δεν ικανοποιεί το αξίωμα $(\langle\boldsymbol{u},\boldsymbol{u}\rangle = 0) \Leftrightarrow (\boldsymbol{u} = \boldsymbol{0})$, αλλά ικανοποιεί όλα τα υπόλοιπα αξιώματα του εσωτερικού γινομένου, τότε ονομάζεται η συνάρτηση αυτή \textbf{ημιεσωτερικό γινόμενο} (semi-inner product).
	
	Παραδείγματα εσωτερικού γινομένου:
	\begin{itemize}
		\item $\mathbb{R}^n$: Για δύο διανύσματα ${\boldsymbol{u} = (a_1, a_2, \ldots, a_n)}$ και ${\boldsymbol{v} = (b_1, b_2, \ldots, b_n)}$, η συνάρτηση $\langle\boldsymbol{u},\boldsymbol{v}\rangle = \sum^n_{i=0}a_ib_i$ ικανοποιεί όλα τα αξιώματα του εσωτερικού γινομένου.
		\item $C[a,b]$: To ολοκλήρωμα ${\langle\boldsymbol{f}(x),\boldsymbol{g}(x)\rangle = \int^b_af^*(x)g(x)dx}$ ικανοποιεί τα αξιώματα της νόρμας και άρα αποτελεί νόρμα του χώρου. Θεωρούμε εδώ ότι οι συναρτήσεις είναι μιγαδικές, οπότε πρέπει να υπολογίσουμε την συζυγή της συνάρτηση στο εσωτερικό γινόμενο.
	\end{itemize}
	
	\begin{definition}
		Ένας χώρος ονομάζεται \textbf{χώρος Hilbert} όταν διαθέτει εσωτερικό γινόμενο και είναι πλήρης.
	\end{definition}

	Όπως και με την νόρμα, δεν είναι μοναδικό το εσωτερικό γινόμενο.
	
	Με την εισαγωγή της συνάρτησης του εσωτερικού γινομένου μπορούμε πλέον να ορίσουμε τις έννοιες του μέτρου και της απόστασης μονοσήμαντα.
	\begin{definition}
		Έστω ανυσματικός χώρος $V$ με εσωτερικό γινόμενο. Ορίζουμε μονοσήμαντα τις συναρτήσεις νόρμας $\|\cdot\|$ και απόστασης $\dist_V(\cdot,\cdot)$ ως εξής:
		\begin{align*}
			\|\boldsymbol{u}\| &= \sqrt{\langle\boldsymbol{u},\boldsymbol{u}\rangle} \\
			\dist_V(\boldsymbol{u},\boldsymbol{v}) &= \|\boldsymbol{u} - \boldsymbol{v}\|
		\end{align*}
	\end{definition}

	Πριν, αναφέραμε πως η έννοια του εσωτερικού γινομένου είναι χρήσιμη ώστε να μπορέσουμε να μιλήσουμε για γωνίες. Μπορούμε να μιλήσουμε, για παράδειγμα, για καθετότητα, ορμόμενοι από το εσωτερικό γινόμενο της τρισδιάστατης Ευκλείδειας γεωμετρίας:
	\begin{definition}
		Ονομάζουμε δύο ανύσματα $\boldsymbol{u}$, $\boldsymbol{v}$ ανυσματικού χώρου $V$ με εσωτερικό γινόμενο, όπου ισχύει $\boldsymbol{u}\neq\boldsymbol{v}$, \textbf{κάθετα} (perpendicular) αν ισχύει:
		\begin{align*}
			\langle\boldsymbol{u},\boldsymbol{v}\rangle = 0
		\end{align*}
	\end{definition}

	Η έννοια αυτή είναι πολύ χρήσιμη για τον ορισμό μιας ιδιαίτερης κατηγορίας βάσης:
	\begin{definition}
		Έστω ανυσματικός χώρος $V$ με βάση ${S = \{\boldsymbol{e}_i\}}$. Θα αποκαλούμε την βάση αυτήν \textbf{ορθοκανονική} (orthogonal and normalized) εάν ισχύει:
		\begin{align*}
			\langle\boldsymbol{e}_i,\boldsymbol{e}_j\rangle = \delta_{ij}
		\end{align*}
	\end{definition}
	Με άλλα λόγια, μια βάση είναι ορθοκανονική αν τα ανύσματά της είναι κάθετα μεταξύ τους αλλά και έχουν μέτρο $\|\boldsymbol{e}_i\|^2 = \langle\boldsymbol{e}_i,\boldsymbol{e}_j\rangle = 1$.
	
	Γιατί όμως να μας νοιάζει μια τέτοια βάση; Έστω άνυσμα $\boldsymbol{u}$. Τότε, στην βάση $S$, η οποία είναι ορθοκανονική, το άνυσμα αυτό θα γραφεί ως:
	\begin{align*}
		\boldsymbol{u} = \sum_{i=0}^{\dim(V)}a_i\boldsymbol{e}_i
	\end{align*}
	Το εσωτερικό γινόμενο του $\boldsymbol{u}$ με ένα άνυσμα της βάσης $S$ θα είναι:
	\begin{align*}
		\langle\boldsymbol{e}_i,\boldsymbol{u}\rangle &= \langle\boldsymbol{e}_i,\sum_{j=0}^{\dim(V)}a_j\boldsymbol{e}_j\rangle  \overset{\text{διγραμμικότητα}}{\Leftrightarrow}\\
		\langle\boldsymbol{e}_i,\boldsymbol{u}\rangle &= \sum_{j=0}^{\dim(V)}\langle\boldsymbol{e}_i,a_j\boldsymbol{e}_j\rangle \overset{\text{διγραμμικότητα}}{\Leftrightarrow}\\
		\langle\boldsymbol{e}_i,\boldsymbol{u}\rangle &= \sum_{j=0}^{\dim(V)}a_j\langle\boldsymbol{e}_i,\boldsymbol{e}_j\rangle \overset{\text{ορισμός}}{\Leftrightarrow}\\
		\langle\boldsymbol{e}_i,\boldsymbol{u}\rangle &= \sum_{j=0}^{\dim(V)}a_j\delta_{ji} = a_i
	\end{align*}
	Έχουμε καταφέρει κάτι πολύ σημαντικό: Έχουμε βρει μέθοδο με την οποία μπορούμε να βρούμε τις συνιστώσες ενός ανύσματος στην ορθοκανονική βάση· αρκεί να υπολογίσουμε το εσωτερικό γινόμενο των ανυσμάτων βάσης και του διανύσματος του οποίου θέλουμε να βρούμε τις συνιστώσες.
	
	Όμως τώρα έχουμε ένα νέο πρόβλημα. Πώς θα βρούμε μια ορθοκανονική βάση; Φαίνεται αρχικά ότι απλά θα πρέπει δια μαγείας να βρούμε μία. Αλλά στην πραγματικότητα, κάθε βάση $S$ μπορεί να γίνει ορθοκανονική.
	\begin{theorem}
		(Μέθοδος Gram-Schmidt) Έστω χώρος $V$ με βάση $S$. Τότε, η βάση $S$ μπορεί να μετασχηματιστεί σε βάση $S'$, η οποία είναι ορθοκανονική, με την αναδρομική σχέση:
		\begin{align*}
			\boldsymbol{e}'_1 &= \frac{\boldsymbol{e}_1}{\|\boldsymbol{e}_1\|} \\
			\boldsymbol{\epsilon}'_n &= \boldsymbol{e}_n - \sum_{i=1}^{n-1}\langle\boldsymbol{e}'_i,\boldsymbol{e}_n\rangle\boldsymbol{e}'_i\\
			\boldsymbol{e}'_n &= \frac{\boldsymbol{\epsilon}'_n}{\|\boldsymbol{\epsilon}'_n\|}
		\end{align*}
		με $n\leq \dim(V)$
	\end{theorem}
	\newpage
	Παράδειγμα: (Πολυώνυμα Legendre)
	Όπως αναφέρθηκε στο πρώτο μέρος της ενότητας των ανυσμάτων, στον ανυσματικό χώρο των συνεχών πραγματικών συναρτήσεων, έστω $C[0,1]$, τα μονώνυμα ${\boldsymbol{p}(x) = x^n, n\in\mathbb{N}}$ είναι βάση του χώρου αυτού. Έστω ότι ορίζουμε το εσωτερικό γινόμενο ως ${\langle\boldsymbol{f}(x),\boldsymbol{g}(x)\rangle= \int^{1}_{-1}f(x)g(x)dx}$. Είναι πασίδηλο ότι η βάση των μονωνύμων ΔΕΝ είναι ορθοκανονική. Π.χ.:
	\begin{align*}
		\|\boldsymbol{1}\| &= 2\\
		\langle x, x^3\rangle &= \frac{2}{5}
	\end{align*}
	Με βάση το ανωτέρω θεώρημα, μπορούμε να την μετατρέψουμε σε ορθοκανονική:
	\begin{align*}
		\boldsymbol{P}_0(x) &= \frac{\boldsymbol{1}}{\|\boldsymbol{1}\|} = 1 \\
		\boldsymbol{P}_1(x) &= \frac{x}{\|\boldsymbol{x}\|} = x \\
		\boldsymbol{P}_2(x) &= \frac{x^2 - \frac{2}{3}}{\|x^2 - \frac{2}{3}\|} = \frac{1}{2}\left(3x^2 - 1\right)\\
		\ldots
	\end{align*}
	Τελικά, καταλήγουμε στην γενική σχέση:
	\begin{align*}
		\boldsymbol{P}_\ell(x) = \frac{1}{2^\ell\ell!}\frac{d^\ell}{dx^\ell}(x^2 - 1)^\ell
	\end{align*}
	Τα ανωτέρω πολυώνυμα ονομάζονται \textbf{πολυώνυμα Legendre} (Legendre polynomials) και είναι ιδιαιτέρως σημαντικά στην Φυσική, καθώς προκύπτουν στην επίλυση διαφορικών εξισώσεων. Τα μονώνυμα $x^n$ είναι γεννήτορες και των πολυωνύμων Laguerre, στον χώρο των πραγματικών συνεχών συναρτήσεων στην ημιευθεία $[0,+\infty)$, και των πολυωνύμων Hermite, στον χώρο των πραγματικών συνεχών συναρτήσεων σε όλο το $\mathbb{R}$. Τα πολυώνυμα αυτά προκύπτουν με διαδικασία πανομοιότυπη με την ανωτέρω, με την εξαίρεση ότι το εσωτερικό γινόμενο των δύο αυτών χώρων είναι ολοκλήρωμα με κατάλληλο βάρος (π.χ. το εκθετικό), έτσι ώστε να μην έχουμε απειρισμό.
	
	Η έννοια του εσωτερικού γινομένου είναι τόσο ισχυρή που αρκεί για να περιγράψει εξ ολοκλήρου έναν ανυσματικό χώρο
	
	\textbf{ΣΗΜΕΙΩΣΗ: Το ακόλουθο εδάφιο δεν είναι απαραίτητο για την κατανόηση των ανυσματικών χώρων και είναι αποκλειστικά για λόγους ενδιαφέροντος.}
	
	\subsection{Περί συντεταγμένων, γενίκευση λογισμού}
	Στο εδάφιο 2.1 κάναμε αναφορά στις συντεταγμένες. Για σύνοψη, συντεταγμένες ονομάζουμε τους συντελεστές του γραμμικού συνδυασμού ενός ανύσματος σε μια βάση την οποία έχουμε διατάξει. 
	
	Στην ανάλυση συναρτήσεων πολλών μεταβλητών, για τον χώρο, π.χ., $\mathbb{R}^3$, καταχρηστικά χρησιμοποιούσαμε τον συμβολισμό $f(x,y,z) = \ldots$ όταν αναφερόμασταν σε μια συνάρτηση της μορφής ${f:\mathbb{R}^3\rightarrow\mathbb{R}}$, για να μπορέσουμε να χρησιμοποιήσουμε τις συντεταγμένες του χώρου. Καταχρηστικά, για τον εξής λόγο:
	\begin{itemize} 
		\item Ο χώρος $\mathbb{R}^3$ δεν «ξέρει» τι θα πει $x$, $y$ και $z$ άξονας. Οι άξονες είναι το αποτέλεσμα επιλογής βάσης του χώρου (ορθοκανονική συγκεκριμένα) και συγκεκριμένα διατεταγμένης. Με άλλα λόγια, οι τιμές $x$, $y$, $z$ είναι οι συντεταγμένες του χώρου και είναι πλήρως εξαρτημένες από την βάση που επιλέξαμε. Από την άλλη, οι τριάδες του διανυσματικού χώρου $\mathbb{R}^3$ μένουν αμετάβλητες. Ατυχώς, ο συμοβλισμός των τριάδων αυτών και ο συμβολισμός των συντεταγμένων συμπίπτουν.
	\end{itemize}

	Τι είναι τελικά όμως αυτά τα $x$, $y$ και $z$; Η απάντηση είναι απλή:
	\begin{definition}
		Για μία διατεταγμένη βάση $S$ ανυσματικού χώρου $V$, διάστασης $\dim(V)$, μπορούμε να ορίσουμε σύνολο συναρτήσεων ${\{e^i:V\rightarrow\mathbb{F}\}}$ τέτοιες ώστε
		να ισχύει:
		\begin{align*}
			e^i(\boldsymbol{u}) = a_i
		\end{align*}
		όπου $a_i$ η $i$-ωστή συνιστώσα του $\boldsymbol{u}$ όταν γραφεί ως γραμμικός συνδυασμός στην βάση $S$. Οι συναρτήσεις αυτές ονομάζονται \textbf{συναρτήσεις συντεταγμένων} (coordinate functions). Την συνιστώσα $a_i$ πλέον θα την συμβολίζουμε πλέον $u^i$. Δηλαδή θα χρησιμοποιούμε το όνομα του ανύσματος και, ως εκθέτη, την ανάλογη συντεταγμένη.
	\end{definition}

	Εφόσον η συνάρτηση $e^i(\boldsymbol{u})$ δίνει πραγματικό αριθμό, βγάζει νόημα να ορίσουμε μία απειρωστή ποσότητα.
	\begin{definition}
		Ορίζουμε ως απειρωστή μεταβολή της $i$-ωστής συντεταγμένης ανύσματος $\boldsymbol{u}$ ανυσματικού χώρου $V$, σε βάση $S$ την ποσότητα:
		\begin{align*}
			du^i \equiv de^i(\boldsymbol{u})
		\end{align*}
		και ως παράγωγο στην $i$-ωστή συνιστώσα του ανύσματος $\boldsymbol{u}$, μίας τυχούσης συνάρτησης ${f:V\rightarrow W}$ (όπου $W$ ανυσματικός χώρος) ως προς την $i$-ωστή συντεταγμένη την ποσότητα:
		\begin{align*}
			\frac{\partial f}{\partial u^i}(\boldsymbol{u}) \equiv \underset{\boldsymbol{v} \rightarrow \boldsymbol{u}}{\lim}\frac{f(\boldsymbol{v}) - f(\boldsymbol{u})}{e^i(\boldsymbol{v}) - e^i(\boldsymbol{u})}
		\end{align*}
	\end{definition}
	Το αντικείμενο $\frac{\partial}{\partial u^i}$ είναι σημαντικό και είναι μια συνάρτηση που παίρνει μία συνάρτηση ${f:V\rightarrow W}$ και την αντιστοιχεί σε μία συνάρτηση ${f':V\rightarrow W}$, που αποκαλείται \textbf{μερική παράγωγος της $f$ ως προς την $i$-ωστή συντεταγμένη}. Αυτό το αντικείμενο ονομάζεται \textbf{γραμμικός τελεστής}. 
\end{document}

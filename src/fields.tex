% Use this file to describe the theoretical background of your project/paper

\documentclass[main.tex]{subfiles}

\begin{document}
	\begin{definition}
		Έστω καλώς ορισμένο σύνολο $\mathbb{F}$. Το σύνολο $\langle\mathbb{F}, +, \times\rangle$ ονομάζεται -αλγεβρικό- σώμα (field) αν οι πράξεις $+$ (την οποία θα την ονομάζουμε \textbf{πρόσθεση}) και $\times$ (την οποία θα ονομάζουμε \textbf{πολλαπλασιασμό}) είναι ορισμένες έτσι ώστε να ικανοποιούν τα κάτωθι αξιώματα:
		\begin{enumerate}
			\item Πρόσθεση:
		 	\begin{itemize}
		 		\item $\forall a, b\in\mathbb{F}\exists! c\in\mathbb{F}(a + b = c)$ (Κλειστότητα)
		 		\item $\forall a, b\in\mathbb{F}(a + b = b + a)$ (Μεταθετικότητα)
		 		\item $\forall a, b, c\in\mathbb{F}[(a + b) + c = a + (b + c)]$ (Προσεταιριστικότητα)
		 		\item $\exists! e_0\in\mathbb{F}\forall a\in\mathbb{F}(a + e_0 = e_0 + a = a)$ (Ουδέτερο στοιχείο πρόσθεσης)
		 		\item $\forall a \in\mathbb{F}\exists!(-a)\in\mathbb{F}[a + (-a) = (-a) + a = e_0]$ (Αντίθετος)
		 	\end{itemize}
		 	\item Πολλαπλασιασμός:
		 	\begin{itemize}
		 		\item $\forall a, b\in\mathbb{F}\exists! c\in\mathbb{F}(a \times b = c)$ (Κλειστότητα)
		 		\item $\forall a, b\in\mathbb{F}(a \times b = b \times a)$ (Μεταθετικότητα)
		 		\item $\forall a, b, c\in\mathbb{F}[(a \times b) \times c = a \times (b \times c)]$ (Προσεταιριστικότητα)
		 		\item $\exists! e_1\in\mathbb{F}\forall a\in\mathbb{F}(a \times e_1 = e_1 \times a = a)$ (Ουδέτερο στοιχείο πολλαπλασιασμού)
		 		\item $\forall a \in\mathbb{F}\exists!a^{-1}\in\mathbb{F}[a \times a^{-1} = a^{-1} \times a = e_1]$ (Αντίστροφος)
		 	\end{itemize}
		\end{enumerate}
	 	
	 	Επιπλέον, και για τις δύο πρέπει να ισχύει το κάτωθι αξίωμα:
	 	\begin{align*}
	 		\forall a, b, c \in \mathbb{F}[c\times(a + b) = c\times a + c\times b] \; \text{(Επιμερισμός)}
	 	\end{align*}
 	\end{definition}
 
 	\begin{remark}
 		Πρέπει για τα ουδέτερα στοιχεία των δύο πράξεων να ισχύει ${e_1 \neq e_0}$.
 	\end{remark}
 	
 	Για συντομία, θα χρησιμοποιείται το σύμβολο του ιδίου του συνόλου του σώματος για να αναφερθεί στο σώμα.
 	
 	Διαισθητικά, τα ανωτέρω σημαίνουν ότι ένα σώμα $\mathbb{F}$ είναι \textit{κάτι} του οποίου  τα στοιχεία του έχουν πράξεις οι οποίες θυμίζουν, στις \textbf{ιδιότητές} τους, αυτές των πραγματικών αριθμών $\mathbb{R}$ (πρόσθεση, πολλαπλασιασμό, αφαίρεση, διαίρεση). Μάλιστα, αυτός είναι ένας τρόπος ορισμού των πραγματικών αριθμών:
 	\textit{Οι πραγματικοί αριθμοί είναι ένα σύνολο $\mathbb{R}$, το οποίο είναι σώμα, είναι διατεταγμένο και κάθε άνω φραγμένο υποσύνολο του έχει ελάχιστο άνω φράγμα στο $\mathbb{R}$.}
 	
 	Δεν είναι απαραίτητο οι πράξεις της πρόσθεσης και του πολλαπλασιασμού σε ένα σώμα $\mathbb{F}$ να είναι πανομοιότυπες με αυτές των πραγματικών αριθμών. Ούτε τα στοιχεία του να είναι αριθμοί. Αρκεί να έχουν ιδιότητες που ορίζονται από τα ανωτέρω αξιώματα.
 	
 	Τα συνηθέστερα παραδείγματα αλγεβρικών σωμάτων είναι οι ρητοί $\mathbb{Q}$, οι πραγματικοί $\mathbb{R}$ και οι μιγαδικοί $\mathbb{C}$.
 	
 	H έννοια του αλγεβρικού σώματος είναι σημαντική έτσι ώστε να μπορέσουμε να ορίσουμε το \textbf{άνυσμα}.
\end{document}
